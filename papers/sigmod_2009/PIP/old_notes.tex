\section{Old Notes}


\subsection{The Representation System}


We extend U-relations \cite{AJKO2008}, which are a class of c-tables
with additional probability distributions. 
There are data value columns typed to have continuously distributed values.
Attribute fields of such columns hold a special attribute value variable
for which we store a pdf in a $W_{\mathrm{continuous}}$ table. We store
these pdfs symbolically.
The local conditions are conjunctions of inequalities; the discrete variables
are only equated with constants; the continuous variables may be used in
inequalities of the form $x<y$ (relating variables) or $x>5$. Note that
equalities involving continuously distributed variables are false; tuples in
which they occur can be
removed. Inequalities using $\neq$ and a continuous variable are true and can
be removed.

We evaluate positive relational algebra as in \cite{AJKO2008}. Just maintaining
the local conditions and checking satisfiability is slightly less
straightforward.

A tuple's local condition is inconsistent if
\begin{itemize}
\item
there are constraints $x \le c$ and $x \ge d$ with $c < d$
where $x$ is a continuous variable,

\item
there is a constraint $x = y$ where $x$ and $y$ are
continuous (by construction, we never equate discrete variables with
variables, only constants),

\item
the atoms over discrete variables are inconsistent (see ICDE paper).
\end{itemize}


\begin{example}
\em
%
For a tuple $(\vec{t}, \phi)$, $\sigma_{A<5}$ returns
\[
\left\{
\begin{array}{lll}
\tuple{\vec{t}, \phi} & \dots & \mbox{$t.A$ is a constant $< 5$} \\
\bot            & \dots & \mbox{$t.A$ is a constant $\ge 5$} \\
\tuple{\vec{t}, (\phi \land x < 5)} & \dots &
   \mbox{$t.A$ is cont.\ variable $x$,
         $x < 5$ consistent with $\phi$} \\
\bot & \dots & \mbox{$t.A$ is cont.\ variable $x$,
                     $x < 5$ inconsistent with $\phi$}
\end{array}
\right.
\]
%
\punto
\end{example}


\subsection{Computing Expectations of Aggregates}


By linearity of expectation, eagg($R$), for relation
\[
R = \{ \tuple{\phi_1, v_1}, \dots, \tuple{\phi_m, v_m} \},
\]
is the sum of the expectations of the individual tuples.
The expectation $E[\tuple{v_i, \phi_i}]$ of a tuple
where
$\phi_i = \bigwedge_j u_j=c_j \land \phi_i'$
and
$\phi_i' =\bigwedge_j x_j \theta x_j' \land
          \bigwedge_j x_j \theta c_j$
(the $u_j$ are discrete and the $x_j, x_j'$ are continuous variables) is
\[
\prod_j \Pr[u_j=c_j] \cdot
\int_{x_1} \cdots \int_{x_k} f_{\phi_i'}(x_1, \dots, x_k) \cdot v_i
\; dx_k \cdots dx_1
\]
where $f_{\phi_i}$ is the pdf of the condition $\phi_i$ and
$v_i$ may (but does not have to) be one of the variables $x_i$.
This can be simplified to
\[
\int_{x_1} \cdots \int_{x_k} f_{\phi_i}(x_1, \dots, x_k) \cdot v_i
\; dx_k \cdots dx_1
\]


If $v_i$ is a constant,
\[
E[\tuple{v_i, \phi_i}] =
   v_i \cdot \int_{x_1} \cdots \int_{x_k} f_{\phi_i}(x_1, \dots, x_k)
\; dx_k \cdots dx_1
\]
may be easier to compute.


If $\phi_i'$ = true, then
\[
E[\tuple{v_i, \phi_i}] =
\prod_j \Pr[u_j=c_j] \cdot
\left\{
\begin{array}{lll}
E[v_i] &\dots& \mbox{$v_i$ is a continuous variable} \\
v_i    &\dots& \mbox{$v_i$ is constant}
\end{array}
\right.
\]
and $E[v_i]$ may be often given with a symbolic pdf
(e.g., for $v_i \sim N(\mu, \sigma)$, $\mu = E[v_i]$).


Usually, however, we will have to compute
$E[\tuple{v_i, \phi_i}]$ by sampling. W.l.o.g., we may assume that
$\phi_i$ does not contain discrete variables.


