
It is immediately plausible that one can do better than re-evaluate a query from scratch whenever the database changes a little. Incremental view maintenance (IVM) capitalizes on this insight \cite{DBLP:journals/tods/BunemanC79,DBLP:conf/sigmod/ShmueliI84,DBLP:conf/sigmod/BlakeleyLT86,roussopoulos-tods:91,DBLP:conf/vldb/CeriW91,DBLP:conf/deductive/GuptaKM92,DBLP:conf/sigmod/GuptaMS93,griffin-sigmod:95,yan-vldb:95,colby-sigmod:96,GHJ1996,kotidis-tods:01}. It is a solid, settled technique that has been implemented in multiple commercial database systems, but has seen little new research activity in recent years and has gathered a little dust.

Now there is an exciting and potentially game-changing new development \cite{ahmad-vldb:09, koch-pods:10, kennedy-ahmad-koch-cidr:11}, an extreme form of IVM where all query evaluation work reduces to adding data (updates or materialized query results) to other materialized query results.  No join processing or anything semantically equivalent happens at any stage of processing. This works for a fragment of SQL with equijoins and aggregation, but without inequality joins or nesting aggregates.


Let us digest this, because the last claim goes counter to query processing intuitions to the point of absurdity. The main idea is the following: Classical IVM revolves around the idea that a materialized view can be maintained under updates by evaluating a so-called delta query and adding its result to the materialized view. The delta query captures how the query result changes in response to a change in the database. The new observation is that the delta query can be materialized and incrementally maintained using the same idea, making use of a delta query to the delta query, which again can be materialized and incrementally maintained, and so on, recursively. This works for classes of queries whose deltas are in some way structurally simpler than the base queries (e.g. having fewer joins), allowing this recursive query transformation to terminate. (It does so with a final trivial $k$-th delta query that does not refer to the database at all.) Termination is ensured for select-project-join queries with certain forms of aggregation, but some other features of SQL (specifically aggregations nested in where-conditions) have to be excluded. 

So where do the joins go? They {\em really} go away, as a benefit of incremental computation. If we want to compute $(x+1)*y$ and know $x*y$ and $y$, we only need to add $y$ to $x*y$, and the multiplication goes away. This is what happens when incrementally maintaining a join, where the join takes the place of multiplication. The pattern just sketched in basic algebra is not just an intuition but exactly what happens, and \cite{koch-pods:10} develops the algebraic framework to formalize this. We observe that the symbol 1 above represents the update workload.  In the incremental query processing framework, it must be a {\em constant} number of tuples that are changed in each incrementation step.

\comment{
To the reader who still cannot accept that joins can be replaced by no joins, we observe that the history of all incremental updates to the materialized view taken together is still essentially an execution of a nested loops join, that is, overall the value $x*y$ is constructed by adding $x$ copies of $y$. So if we put all the work associated with the individual updates happening over time together, the join work is still done. But refreshing the view in response to a single update does not require joins.
}

On paper, this approach clearly dominates classical IVM: if classical IVM is a good idea, then doing it recursively is an even better idea: The same efficiency-improvement argument for incremental maintenance of the base query also applies to the delta query. Argued from the viewpoint that joins are expensive and this approach eliminates them, one should expect a potential for excellent query performance.

But does this expectation translate into real performance gains? A priori, the cost of the bulk addition of materialized views or the costs associated with storing and managing additional auxiliary materialized views (for delta queries) might be more considerable than expected.


\medskip


This paper presents the lessons learned in an effort to realize aggressive IVM as motivated above. It represents an effort spanning nearly three years of intense work, which demonstrates that there are considerable technical challenges to be resolved. These key challenges are described next.

{\bf Compilation of update trigger code.}
%
The work that has to take place to update one materialized view with another (i.e., an auxiliary view representing a delta) is conceptually very simple; it essentially consists of bulk-adding tuple multiplicities of one view to another.

This updating work to be performed is particularly well-behaved and can be exploited for efficient evaluation:
\comment{
As observed in \cite{koch-pods:10}, this work is highly data-parallel. While parallel query evaluation is not the focus of the present paper, the updating work to be performed is particularly well-behaved. This can be exploited for efficient evaluation:
}
Classical query engines employ interpretation and large-gra\-nu\-la\-ri\-ty query operators such as joins to execute query plans.
In the past, IVM has used such query engines to evaluate its delta queries.
Instead, it is natural to avoid both query operators and plan interpretation, and the conceptual simplicity of the required work calls for aggressive code inlining and the elimination of the usual overheads due to interpreted query evaluation. It leads us to the compilation of view refreshing to lightweight machine code.

A considerable challenge is to determine suitable intermediate representations of query expressions to be used in the compiler. Such expressions in general have complex binding patterns which represent information flow. In general, this flow is not exclusively bottom-up.
Examples include complex conditions and nested subqueries correlated with their superqueries. Such expressions have input variables, and can only be evaluated if values for these input variables are given. In general, such expressions have to be materialized, which causes difficulties: how to determine a suitable domain for these input variables for which to materialize the results of the expressions, how to represent and store such materialized structures, and how to dynamically maintain the domains of input variables as updates add previously unseen values.


{\bf Compiler optimizations.}
%
Naively materializing delta queries, their delta queries, and so on causes the materialized views of the higher deltas to have high arity: in fact, the dimensionality can be as high as the arity of the product of all the relations joined together in the input query. The resulting size of the materialized views is of course unacceptable. As observed earlier \cite{ahmad-vldb:09, koch-pods:10} though, the materialized views can be losslessly decomposed into small views: taking a delta each time eliminates some join constraints, turning the views indeed into products.

This calls for factorization and decomposition techniques without which this approach would not be workable. In turn, however, recursive decomposition of delta queries of the various trigger programs (for insertions into and deletions from the relations occurring in the query) may produce large numbers of highly redundant factor views. This makes it essential to aggressively perform common subexpression elimination as well as deforestation and fusion techniques from the compiler literature.

In our implementation and experimentation efforts, this turned out to be much more important for satisfactory performance of the system than expected: without these techniques, recursive compilation for IVM with factorization as described in \cite{koch-pods:10} can result in hundreds of views to be maintained for a large join query, most of which are redundant and can be eliminated.
\comment{
Common subexpression elimination has been studied in the context of multi-query optimization, but here it takes a much more central role: without it, query performance will deteriorate by orders of magnitude almost every time: This is true even though we are referring to the compilation products of a single query, not a workload of multiple queries whose naturalness, if the queries share many subqueries, is often debatable. Thus, optimizations which are typically associated with compilers for general-purpose programming languages rather than query optimizers take a central role.
}
\comment{
We have also learned that some of the natural optimizations used in compilers for general-purpose programming languages rather than query optimizers take a central role. In particular, certain types of common subexpression elimination, loop fusion and analogous deforestation techniques for aggregations, are key to obtaining acceptable performance.
}
However, several of these optimizations such as deforestation have no form of expression in high-level query plan formalisms used in classical query optimization or recursive IVM. Thus, more than one internal intermediate representation (IR) of code is necessary.

\comment{
Our experiences led to two functional followed by one imperative IR. The functional IRs are a cleaned-up form of the algebraic expressions based on rings of queries and databases from \cite{koch-pods:10}, followed by a lower-level, Haskell-like and nearly general functional programming language in which we perform further forms of fusion and deforestation. The imperative IR is used in the compiler backend before imperative code generation. \comment{It is used for further optimizations that cannot be naturally expressed in a functional IR.} Overall, this leads us to a multi-stage reference compiler architecture for optimizing compilation of database queries to imperative code that we believe is general and relevant outside the context of incremental computation (cf. also Delite \cite{delite:11}).
}


\comment{
{\bf Side effects and initial value computation.}
%
Recursive IVM creates challenges we have not seen sufficient study of before, although they occur in simpler form in previous data management architectures that combine updating with querying, such as OLTP systems and stream processors: It is the tension between the convenience of viewing queries as pure functions of the data, and the {\em side-effects} that are updates.
In recursive IVM, an update triggers a variety of computations -- queries -- on various levels of a hierarchy of materialized views; one such computation creates data that is stored and used by another. These interleaved computations conceptually are meant to happen together, and side-effects must be carefully orchestrated to ensure a consistent database state after each update.

\comment{
In the context of recursive IVM, an update triggers a variety of computations -- queries -- on various levels of a hierarchy of materialized views; one such computation creates data that is stored and used by another. Compared to active databases, where certain events can trigger a cascade of computations, in the context of recursive IVM we face additional challenges in that interrelated computations do not profit from the benefits of isolation and conceptual serialization due to transaction semantics. The interleaved computations conceptually are meant to happen together.
}

As mentioned above, we in general need to materialize query expressions with binding patterns: queries that have input variables whose values cannot be determined from the query itself. The dynamic extension of the domains of these input variables and the resulting augmentation of the materialized views requires special initialization code distinct from the incrementation code (the compiled delta queries); this adds additional subtle challenges to the compilation framework. Most importantly, deciding how the domains of input variables of a materialized view has to be extended and optimizing the resulting code requires an intricate analysis of side effects across the hierarchy of materialized views.
} % end comment


{\bf Extraction/materialization as a first-class citizen of query optimizers.}
%
As stated above, the framework of recursive IVM requires delta queries to be structurally simpler than the queries they are deltas to. This is not the case for full SQL. This calls for abstracting from the strict notion of recursive IVM discussed in \cite{ahmad-vldb:09, koch-pods:10}. The key rewriting is to extract and materialize a subquery for IVM. This rewriting can be performed at multiple places in the query, as well as in delta queries that are used to incrementally maintain the materialization. In \cite{ahmad-vldb:09, koch-pods:10} the extracted subquery is always the full query or a factor in its product decomposition. But this is really an arbitrary restriction which can be lifted without causing fundamental problems.

This turns our compilation task into one of generating query evaluation code using an optimizer that has an {\em extract/materialize operator} that can be applied anywhere in the query, subject to optimization decisions (which we know from the literature on answering queries using views) but which also further compiles the delta queries auxiliary to materialization.


\medskip

The structure of the paper is as follows. Next, in Section~\ref{sec:sota}, we provide further motivation for the study of multilevel IVM by demonstrating by experimentation that low-latency continuous queries from ETL, monitoring and computational finance cannot be easily expressed by or managed by classical stream processors, or any other kind of existing engine. 
In Section~\ref{sec:compiler}, we present our compilation approach in detail, covering the various stages of a compiler from multilevel
incremental view maintenance to code generation.
Section~\ref{sec:experiments} returns to experimentation, picking up where we left off at the end of Section~\ref{sec:sota}, with all current data management systems failing to provide satisfactory performance for frequently fresh views for the workload proposed there.
We conclude with Section~\ref{sec:conclusion}.

