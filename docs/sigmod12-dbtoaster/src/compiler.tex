\section{The technical section -- to be split up later}


\subsection{Queries with binding patterns, deltas, and the extract/materialize-incrementally rewriting}

In this section, we formalize an intermediate representation (IR) for SQL queries that is suitable for compilation of database queries and to perform the transformations necessary to enable efficient multilevel incremental view maintenance. This language is a refinement of the query language defined in \cite{koch-pods:10}, but here we aim at better readability and avoid unnecessary formality. We will use SQL syntax wherever we can. The IR is introduced as we discuss its desiderata.

\begin{itemize}
\item
The IR must be small-step compositional, which means that any natural syntactical subexpression must be a valid query and member of the IR. This is in essence true for SQL, if one assigns a natural meaning to subexpressions such as ?A<B? (return 1 if the values of parameters A and B given from outside the query satisfy A<B, and 0 otherwise) and ?from R? (select * from R).

This has two particular implications: First, elements of our IR are queries with binding patterns. Binding patterns represent information flow. In general, this flow is not exclusively bottom-up.
Some of an expression?s variables are input variables or parameters which cannot be computed from the query but have to be given to the query so that it can be evaluated. The most interesting case of this if a correlated nested aggregate is viewed in isolation (which must be possible for small-step compositionality). In such an aggregate, the correlation variable from the outside is such an input variable. The aggregate query can only be computed if a value for the input variable is given.





The second implication is that conditions have to be first-class citizens of the IR and valid queries.

The solution is to think of queries as SQL expressions with binding patterns, i.e., each query $Q$ has input variables or parameters $\vec{x_{in}}$ and a set of output variables $\vec{x_{out}}$, which form the schema of the query result. We denote such an expression as $Q[\vec{x_{in}}][\vec{x_{out}}]$.

Examples. Assume relation schema $R(A,B)$. All of the following expressions are valid queries.
\begin{itemize}
\item
   (select * from R)[][A,B]
\item
   R[][A,B]
\item
   (C<D)[C,D][]
\item
   (select A from R where B < C)[C][A]
\item
   (select * from R r1 where r1.B < (select sum(r2.B) Bsum from R r2 where r1.A > r2.A)[][r1.A, r1.B].
Here, the nested aggregate subexpression has binding pattern [r1.A][Bsum]

\item
For an aggregate query with group-by, this means that the group by columns are the output variables.
\end{itemize}



\item
The language must be closed under taking deltas, i.e. for any query, its delta query must be a member of the language too. This in particular requires us to be able to express deletions as elements of the IR.



Aggregation semantics: tuples with {\em integer} multiplicities, all queries are aggregates.




\item
Aggregate queries dominate analytical workloads and can greatly profit from incremental view maintenance, also because refreshing such query results often just means to update some aggregate values rather than change the table structure (i.e., add or remove tuples). Thus efficiency can profit from a language that is optimized for aggregate query processing. We eliminate the distinction between aggregate and non-aggregate queries.
\end{itemize}
















Explain how joins and combination with elements of the where-clause are multiplications.




\subsubsection{Computing the delta of a query.}

We focus on nested SQL with sum aggs. Select-from-where queries with aggregates and group by, arbitrary nesting. We exclude min/max aggs, null values and outer joins. Explain min/max aggs briefly: deletions need additional storage structures.

Proposition: The query language is compositional and closed under taking deltas.

Any syntactic subquery is a query of the language, and any query has a delta which is a query of the language.



\subsubsection{Recursive incremental view maintenance \cite{koch-pods:10}.}


The query language is a special case without nested aggregation. Termination. No query operators left.



Recursive incremental view maintenance makes use of the simple fact that a delta query is a query too. Thus it can be incrementally maintained as well, making use of a delta query to the delta query, which again can be materialized and incrementally maintained, and so on, recursively.

This works if the following requirement on the query language is satisfied:
Deltas must be structurally simpler than the base queries, allowing this recursive query transformation to terminate. We fix our criterion for greater simplicity to be fewer joins. The recursive transformation terminates for a k-th delta query if it does not contain any database relation.
Termination is ensured for select-project-join queries with certain forms of aggregation, but some other features of SQL (specifically aggregations nested in where-conditions) have to be excluded. 

Example.




\subsubsection{Challenges and insights.}


Explain challenges:
\begin{itemize}
\item
Delta query with input/output vars is a bulk operation: needs to be done for each var assignment to input/output vars. Output vars are essentially group by.

\item
In general, such expressions with binding patterns have to be materialized, which causes difficulties: how to determine a suitable domain for these input variables for which to materialize the results of the expressions, how to represent and store such materialized structures, and how to dynamically maintain the domains of input variables as updates add previously unseen data values.

Maintaining the right domains for the variables is a challenge, particularly for input variables. Great optimization potential and so far we only have a na�ve solution. Assuming all the relevant variables are in the domain, it?s easy, but that is an unrealistic assumption.
\end{itemize}



We learn two lessons:
\begin{itemize}
\item
This calls for a compilers approach. Since the work gets conceptually so simple and well-behaved, we should be able to gain from doing something different from what classical query interpreters do. We have to study to what extent this is true if we return to the larger query language with nested aggs.

\item
There is a need for decomposition and factorization to make this useful. Next section.
\end{itemize}



















\begin{verbatim}
select A, sum(C) from R, S where R.B<S.B group by R.A

q = (select sum(C) from R,S where R.B<S.B)[][R.A] 

+R(a,b):
   q[][a] += q1[b][]

q1 = (select sum(C) from S where y<S.B)[y][]

+S(b,c):
   foreach y: q1[y][] += select c where y < b


(y<b) ? c : 0

q[\vec{v}_b][\vec{v}_f]
\end{verbatim}




Use SQL with input-output signature. Every query is an aggregate, we maintain multiplicities. This is important for a clean handling of deletions, but it is optimized for aggregate-group-by queries.

\begin{verbatim}
(x)->(A)-> sum_B(R *  C=x)
select A, sum(B) from R where C=x group by A.


(x) -> (A)-> {0,1}
select A from R where C=x
\end{verbatim}


Technically, rather than computing tuples we compute multiplicities.


Table for delta rules: use relational algebra notation.

Delta under single-tuple update of a query for a particular relation. Creating update trigger programs.

Give example for an aggressive compilation.





\subsection{Join graph decomposition and factorization}


\begin{verbatim}
R(AB), S(BC), T(CDE)

select sum(A*D) from natjoin(R,S,T) [][T.E]


+S(b,c):
   foreach x: q[][x] += select sum(A*D) from R,T [][E]
         = select sum(A) from R where B=b [][] *
            select sum(D) from T where C=c [][T.E/x]
\end{verbatim}


Factorization:

\begin{verbatim}
R(A,B)
q = select sum(r1.B) from R r1, R r2 where r1.A = r2.A [][r2.B]  
\end{verbatim}


three terms; 





\subsection{Initial value computation}


Theorem: case where values are zero.

Compile initial value computations for incremental evaluation. This needs non-query code.





\subsection{Inequality Joins and nested aggregation}

Bigsum rewriting: Oliver


\begin{verbatim}
R(ABC)
Q = select sum(r1.A) from R r1 where
   5*r1.B < select sum(r2.A) from R r2 where r1.C=r2.C [][]

q2 = select sum(A) from R where x=C[x][]

+R(a,b,c):
foreach x: q2[x][] +=  
\end{verbatim}




Choice of subquery extraction / common subexpression elimination.





\subsection{Overall compiler architecture}

\subsection{The functional compiler/optimizer}

Yanif

K3 optimizations: what cannot be expressed on the level of relational algebra; optimization rules. 

\subsection{Code generators and runtime system}

Yanif/Oliver





