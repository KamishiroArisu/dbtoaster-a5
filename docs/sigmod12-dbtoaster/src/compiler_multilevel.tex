\subsection{Extraction Heuristics}
\label{sec:advanced-rewriting}

This subsection gives heuristics for choosing which subexpressions of
a query expression to extract and materialize for incremental view maintenance.  These heuristics apply to user-provided queries, as well as delta queries created by the compilation algorithm of Section~\ref{sec:compiler_calc} -- we have observed experimentally (see Section~\ref{sec:experiments}) that under certain conditions, less aggressive incremental maintenance can be beneficial.

We express these conditions in terms of five heuristics:

\tinysection{Do not incrementally maintain cross-products or similar expressions}

The first heuristic revisits the idea of query decomposition.  Iteration over a cross product can not be made faster through materialization.  The same idea extends to joins bearing similarity to cross products: equi-joins that follow foreign key dependencies (e.g., {\tt ORDERS} $\bowtie_{\tt custkey}$ {\tt CUSTOMER} in the TPC-H schema, with {\tt orderkey} fixed) or inequality predicates.

\tinysection{Keep nested aggregate queries materialized separately}

Aggregate subqueries nested inside comparison operators are a a challenge for the delta operator -- the delta of an aggregate expression is not strictly simpler than the original expression.  However, the complexity of the delta term is limited to two exact copies of the original expression -- apart from these, the delta expression is actually simpler.  Materializing the original aggregate expression reduces the cost of evaluating the nested aggregate to the cost of a join over any correlated terms.

\tinysection{Do not incrementally maintain views with both input and output variables}

The work an insertion performs to maintain an output-only view is limited by the expression defining the delta of the view.  Pushing aggregation down into the materialized delta reduces this bound even further.  The lower-level the delta, the simpler the expression, and the less work is involved in maintaining it.

Conversely, maintaining a materialized view with input variables requires iterating over a domain of values defined outside of the map, with that domain typically increasing monotonically as the query runs -- the amount of maintenance work required does not decrease for lower and lower levels of deltas.  As long as the generated datastructure stays small, this is ok -- we have found performance to be for views with only input variables.  In general, predicates introducing input variables should be applied as filters after the insertion.

\tinysection{Do not materialize views for the maintenance of base relations that are never updated}

As an example, the NATION table in the TPC-H schema consists of 25 nations, and is never updated (or updated infrequently at best).  Materializing views for the maintenance of this relation creates extra, unnecessary work.

\tinysection{Avoid performing 3-way (or wider) equi-joins at evaluation time}

In order to conserve memory by materializing fewer views, we have considered limiting the recursive compilation in our compiler to a fixed depth (discussed in further depth in Section \ref{sec:experiments:othermetrics}).  The result is fewer materialized views, but wider joins required for view maintenance.  We have found that 3-way joins are prohibitively expensive -- performance is worse even than using traditional IVM techniques.