\section{Compiler extensions}
\label{sec:extensions}

\subsection{Min/Max aggregates}
\begin{itemize}
\item Aggregate rules for min/max queries.
\item Describe the need to maintain relational state for handling delete min/max.
\item Describe the issues with nesting min/max aggregates with existing map
  algebra terms.
\end{itemize}

\subsection{Monotonic nested queries}
\begin{itemize}
\item Describe potential to keep only a limited subset of variables defined from
  outer queries, which in the extreme case of monotonic queries simplifies down
  to a single value.
\item Describe data structures enabling initial values to be computed from
  the nested query result of a neighbouring outer variable value.
\end{itemize}

\subsection{Compiling multiple queries}
\begin{itemize}
\item Currently support multiple aggregates specified on top of the same
  relational statement, which may subsequently share maps if there are any
  equivalent map algebra expressions found common during compilation.
\item We partition queries with common (but not an equivalent set of) input
  relations into compilation units, enabling handler functions to be created for
  each compilation unit. This prevents monolithic event handlers, thus our
  compiled database engine may call multiple handlers for a single update stream.
\end{itemize}
