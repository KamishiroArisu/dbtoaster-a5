

Both the computation of moments and probabilities in the general case reduces to numerical integration, and a dominant technique for doing this is Monte Carlo simulation. The approximate computation of expectation
\begin{equation}
E[\chi_\phi \cdot (h \circ t)] =
\frac{1}{n} \cdot \sum_{i=1}^n p(\vec{x}_i) \cdot \chi_\phi(\vec{x}_i) \cdot
h(t(\vec{x}_i))
\end{equation}
faces a number of difficulties.  In particular, samples for which $\chi_{\phi}$ is zero do not contribute to an expectation.  If $\phi$ is a very selective condition, most samples do not contribute to the summation computation of the approximate expectation.  (This is closely related to the most prominent problem in online aggregation systems \cite{OnlineAggregation,DBO}, and also in MCDB).

This section describes how we address these issues in PIP.

\label{subsec:csampling}
The primary challenge faced by PIP is that - particularly in highly selective queries - the general-purpose sampling routines provided for each distribution will produce samples that violate the query constraints $\phi$.  (This is closely related to the most prominent problem in online aggregation systems \cite{OnlineAggregation,DBO}, and also in MCDB).

For example, consider a row containing the variable $Y \sim Normal(5,10)$ and the condition predicate $(Y > -3)$ and $(Y < 2)$.  The expectation of the variable $Y$ in the context of this row is not $5$.  Rather the expectation is taken only over samples of $Y$ that fall in the range $(-3,2)$.

{\bf Rejection Sampling}.
One straightforward approach to this problem is to perform rejection sampling; sample sets are repeatedly generated until a sufficient number of viable (satisfying) samples have been obtained.  Note that this is not strict rejection sampling, samples for with $\chi_{\phi}$ is zero count towards the number or samples $n$ by which we average.  

However, without scaling the number of samples taken based on $E[\chi_\phi]$, information can get very sparse and the approximate expectations will have a high relative error.  Unfortunately, as the probability of satisfying the constraint drops, the work required to produce a viable sample increases.   Consequently, any mechanism that can improve the chances of satisfying the constraint is beneficial.

{\bf Sampling using inverse CDFs}.
As an alternative to sampling using the generator function, PIP can use the inverse-CDF of a distribution (if available) to translate a uniform random number in the range $[0,1]$ to a variable sampled from the distribution.  The CDF increases monotonically, so to obtain samples constrained to $[lower, upper]$, we compute $CDF^{-1}(X)$ where X is chosen uniformly  from the range $(CDF(lower), CDF(upper))$.   

In the event that the inverse CDF is available, but the CDF is not, this technique may still be used.  Instead of sampling from the range $(CDF(lower), CDF(upper))$, we sample $x \in (L,H)$ where $L$ is initialized to $0$, and $H$ is initialized to $1$.  If $CDF^{-1}(x) \leq lower$ we set $L = x$ and try again.  Similarly, if $CDF^{-1}(x)  \geq upper$ we set $H = x$ and try again.   In this way, PIP effectively learns the values of $CDF(lower)$ and $CDF(upper)$.  

If precise bounds can be derived from the constraints on a given variable, this process guarantees that each sample generated will satisfy the constraint.  Even if only weak bounds are available, the process still provides a benefit.  By reducing the size of the sampling area, the probability of selecting a viable sample is still increased.  

{\bf Exploiting independence}.
Viable samples are also encountered with higher frequency when the number of variables being constrained is smaller.  Fewer variables mean less work is lost when a sample is rejected.  Similarly, fewer constraints mean a lower overall chance of rejection.  PIP exploits this by first subdividing constraint predicates into minimal independent subsets.  Two constraint subsets are independent if their member predicates have no variables in common.  When determining subset independence, composite random variables (for instance, defined by artithmetic expressions over random variables) are treated as the set of all of their component variables.  Note that two variables may appear in the expectation function and still be considered independent; only the selection predicates are considered when determining independence.

By definition, variables in each set are independent.  Thus, the probability of each subset may be computed independently as well; the overall probability is the product of the independent probabilities.  For example, consider the one row c-table 
\[
\begin{tabular}{c|c}
R & $\phi_2$ \\
\hline
& $(X > 4) \wedge ([X\cdot Z] > Y) \wedge (A < 6)$ \\
\end{tabular}
\]
In this case, the atoms $(X > 4)$ and $([X\cdot Z] > Y)$ form one minimal independent subset, while $(A < 6)$ forms another.

Condition atoms describing discrete variables are all of the form $Var = Val$, discrete variables are handled trivially.  Inconsistent values have already been removed, so the probability for the entire subgroup is the probability of the listed variable assignment.

The simplest subset of continuous atoms is one that references only one variable.  In this case, the atoms in the set provide constant upper or lower bounds, and the integration problem may be solved by evaluating the variable's CDF at the tightest upper and lower bounds.  If the CDF is not available or if it is not possible to derive tight bounds on the CDF, PIP can still integrate via Monte Carlo sampling.  

{\bf Metropolis}.
A final alternative available to PIP, is the
Metropolis  algorithm \cite{metropolis}.   Starting from  an arbitrary
point within the sample space,  this algorithm performs a random walk.
Steps  are  sampled  from  a  multivariate  normal  distribution,  and
rejection sampling  is used  to weight the  walk towards  regions with
higher  probability  densities.  Samples  taken  at regular  intervals
during the random walk may be used as samples of the distribution.

The Metropolis algorithm has an  expensive startup cost, as there is a
lengthy  `burn-in' period  while  it generates  a sufficiently  random
initial  value.  Despite  this startup  cost, the  algorithm typically
requires only a relatively small  number of steps between each sample.
Consequently,   the  Metropolis  algorithm   is  ideally   suited  for
generating large numbers of samples  when the CDF is not available and
the probability of sampling a given value is small.

%{\bf Integration}
%
%There are  cases where bounds are insufficient.   For example, concave
%atoms  are   not  likely   to  admit  effective   rectangular  bounds.
%Similarly, even though a bounding box  covers no less than half of the
%volume of a contiguous convex constraint area, the  bulk of the  probability mass may
%still lie  outside of the constrained  sample area.  In  such cases, a
%recursive technique may be applied.
%
%The  bounding box  is first  subdivided into  smaller  regions.  Monte
%Carlo integration  is performed on the  region twice, but  with only a
%small number of iterations apiece.  If the two results agree to within
%the  desired  precision, integration  stops  and  the  average of  the
%results is multiplied by the  probability of a sample falling into the
%sampling region.  If the  two results differ significantly, the region
%is further subdivided and the algorithm recurses on each sub-region.
%
%Because the  recursive cutoff is determined by  the estimated accuracy
%of the  result, this  algorithm will not  recurse on regions  that are
%entirely   within  or   outside  of   the  constrained   sample  area.
%Consequently, the majority of  samples generated by the algorithm will
%be  put towards estimating  relatively high  values where  Monte Carlo
%integration is most effective.



