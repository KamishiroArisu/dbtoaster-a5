\documentclass[11pt]{amsart}
\usepackage{geometry}                % See geometry.pdf to learn the layout options. There are lots.
\geometry{letterpaper}                   % ... or a4paper or a5paper or ... 
%\geometry{landscape}                % Activate for for rotated page geometry
%\usepackage[parfill]{parskip}    % Activate to begin paragraphs with an empty line rather than an indent
\usepackage{graphicx}
\usepackage{amssymb}
\usepackage{epstopdf}
\DeclareGraphicsRule{.tif}{png}{.png}{`convert #1 `dirname #1`/`basename #1 .tif`.png}

\title{Alpha 5 Compilation}
\author{Oliver Kennedy}

\begin{document}
\maketitle

\section{Compiler Work Loop}

The compiler maintains a work queue datastructure.  As long as the work queue is non-empty, the compiler will perform the following steps on whatever expression is at the head of the work queue\footnote{This section assumes that the head is a standard map datastructure.  If not, then we might do something more interesting.  We'll find out once we actually start supporting non-map datastructures.}:

\begin{enumerate}
\item {\bf Optimize the expression} The expression is simplified and factorized as discussed below.

\item {\bf Prematerialize the expression}.  For every Lift term that appears in the expression, a copy of the subexpression being lifted is set aside and stored for use in the materialization stage.  

\item {\bf Subdivide the expression} (optional) In the future, we may wish to subdivide the expression into smaller chunks, each of which (after we compute deltas) is guaranteed to be part of a separate materialized subexpression.  For now, this is ignored.

\item {\bf Compute Deltas}.  For each {\em stream} appearing in the expression, apply the delta operation to the expression to get a set of delta expressions.

\item {\bf Materialize the delta expressions}.  Subexpressions are pulled out and replaced with Externals.  This process is discussed in detail below.

\item {\bf Update the work queue}.  Newly instantiated datastructures are added to the work queue.  The recently compiled datastructure is also stored.
\end{enumerate}

\subsection{Expression Optimization}

Optimization provides the following guarantees (each being a rewrite rule):
\begin{enumerate}
\item Values appearing in sum and product terms are merged together and constants are combined/curried.
\item Equality comparisons are converted into Lifts as much as possible.
\item Lifts are unified as much as possible.
\item Lifts and AggSums are un-nested as far as possible: Expressions that always evaluate to 0 or 1 (comparisons, lifts, and keyed relations) can always be pulled out of a lifted product.  If all of the output variables of an expression are group-by variables of an AggSum of a product that the expression appears in, the expression can be lifted out.  These variables can then be removed from the AggSum's group-by variable list.
\item AggSums are factorized: AggSum([\ldots], R(A) * S(B)) becomes AggSum([\ldots], R(A)) * AggSum([\ldots], S(B))
\item Unnecessary AggSums are eliminated if the aggregate does not project away any variables (i.e., if the group-by variables of the AggSum are identical to the aggregated expression's output variables).
\end{enumerate}

\section{Expression Materialization}

Materialization partitions an expression into multiple subexpressions, each of which is materialized as a datastructure.  

\section{Domain Maintenance}



\end{document}