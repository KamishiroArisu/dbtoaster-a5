\section{Introduction}





Modern relational databases  focus almost exclusively
on providing flexible and  extensible querying capabilities in dynamic
environments, supporting  both schema and data  evolution.
As a consequence, today's database management systems are centered around
highly  flexible interactive query processors, with their
plan  interpreters and other  runtime components  including schedulers
and optimizers.   However, a
large fraction of  the world's query workloads are  fixed and embedded
into database application programs. In these applications, queries are
\textit{ossified}  during  the   development  cycle,  with  developers
converging to a choice of schema, query structure and parameters. Once
hardened,  queries  are  deployed  into production  environments,  and
re\-used numerous times, executing non-interactively.



In this paper, we present \compiler,
a novel approach to compiling SQL aggregate queries into
efficient C++ code for  continuous, incremental view maintainance.  
\compiler\ is an SQL query compilation framework which generates native code
that incrementally  and continuously maintains aggregate views in main memory
at high update rates using aggressive delta processing techniques.

Queries are {\em recursively}\/ compiled into incremental view maintenance
code; that is, while increments to queries are traditionally again expressed and evaluated as que\-ries, we recursively compute increments to these increments, and so on. This usually allows us to completely eliminate all queries.
It is, however, not an just an exercise in thorough compilation, but usually
yields code that is substantially simpler than the code needed for query processing in previous incremental view maintenance approaches.

\compiler\ provides three key advantages.
\begin{itemize}
\item
By generating C++  code that  performs all the  processing required  for an
input  tuple from a fixed query plan execution path, we avoid
overheads that traditionally arise when query processors interpret
query plans stored in dynamic data structures.

\item
By recursive compilation, asymptotically faster code can be obtained because
the computation of increments may decompose into code that avoids database
scans or joins.
%
In addition, certain redundancies in query definitions are exposed
and very easily eliminated that have traditionally been considered too difficult to detect and eliminate by query optimization techniques.

\item
Our new delta processing techniques are designed specifically for
compilation to native code and support aggressive inlining and other optimizations performed by modern compilers which lead to
surprisingly small and simple straight-line code sequences.
\end{itemize}



Our query compilation framework is based on the notion of composing maps
definable in a {\em map algebra}\/ closely motivated by SQL.
We compile insert, update, and delete operations into efficient code for computing
deltas of these maps. This corresponds to a form of
tuple-at-a-time processing in a query plan, but can be highly optimized by
modern C++ compilers to yield an efficient native binary for query execution.
Given the inherent streaming nature of the applications we consider, the maps we
maintain during query execution incur a different footprint compared to both
main-memory databases and stream processing engines, since in many cases we need
not keep around base relations, but instead maintain precomputed
views.



\begin{example}\em
Consider a relational database with two relations or schema $R(A,B)$ and
$S(C,D)$, and the aggregation query
\begin{verbatim}
select sum(R.A * S.D) from R, S where R.B = S.C;
\end{verbatim}
This query is admittedly a little artificial but will serve us to demonstrate
some of the main ideas of our approach. More interesting examples will be
compiled later.

In our approach, we compile such a query recursively
into insert and delete event triggers.
It is easy to verify that, assuming that variable $q$ stores the query result
for the current database, we can update $q$ to the correct query result on the
insertion of tuple $(a,b)$ into $R$ by incrementing it by $a * q_D[b]$,
where $q_D[b]$ is the result of the SQL query
\begin{verbatim}
select sum(D) from S where C = b;
\end{verbatim}
Similarly, we can maintain $q$ under insertion of tuple $(c,d)$
into $S$ by adding $q_A[c] * d$ to $q$, where $q_A[c]$ is the result of
query
\begin{verbatim}
select sum(A) from R where B = c;
\end{verbatim}
Now, rather than stopping here (which would be in line with more traditional
incremental view maintenance mechanisms), we recursively compute, and maintain,
the increments to the increment queries $q_A[.]$ and $q_D[.]$.
Here, on insertion of tuple $(a,b)$ into $R$, $q_A[b]$ is incremented by $a$
and on insertion of tuple $(c,d)$ into $S$, $q_D[c]$ is incremented by $d$.
In this case, the recursive computation of increments allows to eliminate
all SQL code and leads to extremely efficient straight-line code:
\begin{verbatim}
on initialization do
   q := 0;

on insert into R values (a,b) do
{
   q += a * q_D[b];

   if q_A[b] is undefined then q_A[b] := 0;
   q_A[b] += a;
}

on insert into S values (c,d) do
{
   q += q_A[c] * d;

   if q_D[c] is undefined then q_D[c] := 0;
   q_D[c] += d;
}

on delete of (a,b) from R do
{
   q -= a * q_D[b];
   q_A[b] -= a;
}

on delete of (c,d) from S do
{
   q -= q_A[c] * d;
   q_D[c] -= d;
}
\end{verbatim}
Here, $q_A[.]$ and $q_D[.]$ are auxiliary associative array data structures.
It is not hard to verify that this code for incrementally maintaining $q$ is
correct.

Note that this code follows the standard default semantics of SQL queries of
not performing duplicate elimination. Moreover, it is assumed that
only tuples are deleted that were inserted first (which can be checked,
if needed, by efficient code and data structures that are independent from
the aggregate computation performed here).

It is worth noting that this code is straight-line and all update
triggers can be executed in constant time. Traditional incremental view
maintenance mechanisms would usually attempt to evaluate the auxiliary queries
$q_A[.]$ and $q_D[.]$ to compute an increment to $q$,
which in general is more costly.
We precompute these auxiliary queries, and make this efficient by performing
incremental view maintenance for $q_A[.]$ and $q_D[.]$ at the same time as
we perform incremental maintenance of $q$.
\end{example}



\subsection{Applications}


\compiler\ is particularly well suited in continuous query applications
where a view has to be maintained through high data update rates. \compiler\ is
motivated  by  applications  that  require  the  highly efficient answering of
fixed workloads of aggregation queries, such as in  data stream  processing,
online data  warehouse  loading, and  in financial applications. Our work on
\compiler\ was particularly motivated by the need for query processing support in
algorithmic equities trading, which in our view cannot be efficiently served by
any of the data management tools available today, as well as aggregate processing
in data stream systems and data warehouse loaders.


\medskip


\textbf{Processing order books in equities trading.}
Following a call for greater transparency~\cite{sec-orderbook:00} earlier this
decade, many stock exchanges provide investors with complete bid and ask limit
order books, enabling a superior view of the market microstructure for use in
trading algorithms. The bid order book consists of prices and volumes of orders
of investors who are willing to buy equities in descending price and timestamp
order, and correspondingly the ask order book indicates investors' selling
orders. Exchanges execute trades by matching the tops of the bid and ask order
books. Investors continually add, modify or withdraw limit orders, thus we view
order books as relations subject to high volumes of order inserts, updates and
deletes.

Investors and automated trading systems express diverse trading strategies on
these order books, and the success of trading depends critically on the speed at
which the programmed strategies process the data.
In fact, the availability of order book data has yielded substantial
opportunities for automatic, algorithmic trading approaches, and in recent years,
algorithmic trading systems have come to account for a majority of volume traded
at the major US and European stock markets.

Unlike stream processing scenarios \cite{abadi-vldbj:03,motwani-cidr:03}, order
books do not grow unboundedly in practice, but cannot be expressed by windows
given their arbitrary inserts, updates and deletes. Thus in \compiler, we process
continuous queries over temporal snapshots of relations via delta processing.
Providing such aggregate views allows {\em algos}\/
to run sophisticated trading
strategies.
\comment{
However in the order book scenario, neither time-, row- or punctuation-based
window semantics apply, rather each insert, update or delete statement defines
a new temporal snapshot of the order book relation. This data model is
particularly relevant since trading strategies continuously query the full
order book to determine an action.

However, trading strategies continuously require access to the full order book
relation, which cannot be expressed by windows given arbitrary inserts, updates
and deletes. Furthermore, order books do not grow unboundedly in practice, the
number of updates and deletes is proportional to the number of inserts and there
is no need for the system to manage the scope of the relation. Thus in
\compiler, we process continuous queries over temporal snapshots of relations
expressed as deltas to base relations.

High-level declarative languages have been applied in similar areas of finance,
such as technical analysis, for example through stream processing engines
\cite[abadi-vldbj:03,motwani-cidr:03]. However, trading strategies continuously
require the full order book and order books do not grow unboundedly in practice
-- they cannot be expressed by windows given arbitrary inserts, updates and
deletes. Thus in \compiler, we process continuous queries over temporal snapshots
of relations expressed as deltas to base relations.
}
To illustrate this, we provide a simple example query used in the popular Static
Order Book Imbalance (SOBI) trading strategy. SOBI computes a volume-weighted
average price (VWAP) over those orders whose volume makes up a fixed upper
{\tt k}-fraction of the total stock volume in each of the bid and ask order books. SOBI
then compares the two VWAPs. For simplicity, we present the VWAP for the bids
only:

\begin{verbatim}
select avg(b2.price * b2.volume) as bid_vwap
from   bids b2
where  k * (select sum(volume) from bids)
         > (select sum(volume) from bids b1
            where b1.price > b2.price);
\end{verbatim}
%    as vwap_input
%
Above, the contents of the bids order book is continually changing,
and \compiler\ produces a new output on every insert, update or delete.


\comment{
Above, the total volume in the order book is continually changing, thus computing
the VWAP over those orders comprising the top $k$\% of volume requires the full
order book relation. This is easily expressed with SQL over standard relations.
\compiler\ produces a new output on every insert, update or delete to the order
book.
}
\comment{
\begin{verbatim}
select case
    when s.vwap - b.vwap > threshold then
        'buy', V, lastPrice - delta, 'sell', V, lastPrice + hedgeDelta
    when b.vwap - s.vwap > threshold then
        'sell', V, lastPrice + delta, 'buy', V, lastPrice - hedgeDelta
from
     (select avg(bcv.price*bcv.volume) as vwap from
        (select sum(volume) as total from bids) as bv,
        (select b2.price, b2.volume, sum(b1.volume) as cumsum from
            bids b1, bids b2 where b1.price > b2.price group by b2.price, b2.volume) as bcv
        where bcv.cumsum < k * bv.total) as b,
    (select avg(scv.price*scv.volume) as vwap from
        (select sum(volume) as total from asks) as sv,
        (select s2.price, s2.volume, sum(s1.volume) as cumsum from
        asks s1, asks s2 where s1.price < s2.price group by s2.price, s2.volume) as scv
    where scv.cumsum < k * sv.total) as s
\end{verbatim}
}
\comment{ Give a nice example of a trading strategy that delegates the bulk of
the work to compiled SQL queries. This is a nice scenario because order books do
not get very large (if we want to process several order books, we can nicely
partition the work by order book across several machines), change extremely
rapidly, and the success of trading depends crucially on the speed with which the
programmed strategies process the data. }


\comment{
\textbf{Stream processing with complex, nested aggregates.}
Previous work on data stream processing and incremental view maintenance has
mostly neglected supporting the efficient processing of complex nested
aggregates. However, such queries do have important applications. For example,
the VWAP query above is highly challenging and, to the best of our knowledge,
cannot be efficiently processed by any of these systems.
}
\comment{
(Note that the above
VWAP applies to the order book, which differs from the technical analysis method
which computes a VWAP over a short time-based window.)
}


\textbf{Data warehouse loading.}
Loading large data warehouses is a computationally costly process, which
causes most data warehouse loading to be performed offline.
While commercial warehouse loaders use specialized efficient code for 
aggregation, incoming data is often the result of data integration
queries that are costly and inefficient, and which may blow up data sizes
in such a way that loading remains inefficient.
Compiling data integration and aggregation queries together may yield efficient
code for loading the warehouse which may avoid the materialization of large
intermediate results.


\textbf{Complex event processing, real-time business
intelligence and data stream analytics.}
The need to keep pace with the growing volume of logistic and operational data
has encouraged businesses to adopt high throughput oriented data processing
systems. While stream and complex event processing engines have endeavoured to
fill this need, they fundamentally require significant retooling of existing
in-house application logic to fit under the hood of data stream management
systems. By generating an in-memory query processor, our compilation framework is
capable of exposing data management functionality to client applications,
enabling clients to seamlessly embed query processing techniques into existing
infrastructure including messaging systems and highly specialized analytical
processing. Furthermore compilation presents significant opportunities for
improving the efficiency and throughput of such applications by supporting a
deeper analysis of both client and data processing functionality, that can
subsequently be leveraged to optimize the whole application.

\medskip



\nop{
We focus on applications that do not require access to large static
databases, rather our applications access bounded
size relations, where the contents of these relations change frequently through
inserts, updates and deletes, in streaming fashion. We assume a single entity
conveys these updates to the database, this is not a multi-user
highly-concurrent
production environment as found in OLTP applications.
}



\subsection{Contributions}


\comment{
We analyse the memory utilization of the data structures resulting from
our decomposition. Additionally, we extend the functionality of our compiler
beyond that of the compiling a single query, including features such as optimized
execution for bulk operations based on loop optimization techniques, multi-query
compilation that can share data structures given commonality amongst queries, and
a limited form of query flexibility that allows users to change parameters during
query execution through parameterized compilation.
}


As we  show, our techniques are several orders of  magnitude   faster  than
the state of the art,  and significantly outperform stream processing
engines on such workloads.  In the case of queries on limit order  book  data  as
 required for  supporting  algorithmic  equities trading, our approach currently
stands alone in its ability to support realistic  data rates  on contemporary
hardware without  resorting to very substantial computing clusters. Indeed, the
memory consumption of our main-memory techniques is sufficiently low to support
applications such as data warehouse loading. Moreover, in most of these
applications, our delta processing techniques, which continuously maintain a
current view of the query result, outperform batch processing techniques even if
we only want to access the view once. That is, our incremental query result
construction usually outperforms one-time query evaluation with traditional
techniques.


We summarize our contributions as follows:

\begin{enumerate}
  \item We present a novel compiler for SQL aggregate queries
  that produces C++ code containing
  straight-line functions for processing tuple inserts, deletes and updates
  along a path through the query plan. Modern C++ compilers can aggressively
  optimize this function allowing our query executor to eliminate significant
  query interpretation overhead.
  \item We present a map algebra as the foundations for producing our
  tuple-processing functions, where map expressions can contain parameterized
  aggregates. The map algebra can easily represent operations on in-memory
  hashtables that are directly used to implement SQL group-by aggregations in
  main-memory data\-ba\-ses. Through our map algebra, we present a compilation
  algorithm that applies rewrites to map algebra expressions, leaving them in a
  form that contains no relational operations, allowing for straightforward
  generation of procedural code.
  \item We experimentally demonstrate that \compiler\ is able to produce
  query executors whose performance significantly dominates existing relational
  database and stream processing engines across a variety of applications. 
\end{enumerate}


This paper is laid out as follows. Section 2 describes the role of the compiler,
and a high-level view of \compiler, a new project at Cornell investigating
database engine runtimes using executors custom compiled for repeated and
standing queries.
\comment{
As part of its scope, \compiler\ will investigate scaling
compiled queries over both multi-core and distributed main-memory databases.
}
\comment{
Section 3 describes our map algebra that enables query transformations
to both select the maps we maintain during execution, and derive the maintenance tasks
for each map. We also describe structural query decomposition techniques for
minimizing the memory consumption of the internal data structures need by our
techniques.
}
Section 3 describes our map algebra that picks map to materialize, and a
structural query decomposition technique for minimizing the memory consumption of
these maps. Section 4 describes the aforementioned extensions to the compiler
for bulk operations and lazy evaluation. Finally prior to discussing
related work and concluding, Section 5 presents experimental results
demonstrating the significant benefits of our techniques over a standard
relational database, a streaming engine, as well as a direct compilation of the
query plan as a straight-line function.






\comment{
We compile queries into C programs. Queries are incrementally maintained.
We are smart about creating data structures for very efficiently keeping query
results up to date. The programming interface supports update triggers as
well as cursor-based iteration over the current query result.

Tasks:
\begin{itemize}
\item
Compile multiple queries into common code, share data structures.

\item
How to compile queries of which some selection conditions can be modified at
runtime? (As in ODBC prepare statements.)

\item
Scale up using data partitioning, parallelization. This is not the focus of
this paper.

\item
Should we say something about accessing secondary storage, or will we leave
this out of scope? If we are not restricted to streaming scenarios, we may
have broader applicability of our techniques.

\item
We have to analyse our main memory consumption.
\end{itemize}


Aim: query processing where the queries are fixed at compile time.
In particular applications where there is no large static database of which
only a small part needs to be processed. The approach works particularly
well if the data arrives on a stream or changes very rapidly

We assume that there is only a single user who performs updates, and that
we do not care about concurrency control.
This is not an OLTP scenario.

Applications:
\begin{itemize}
\item
Processing order books. Give a nice example of a trading strategy that
delegates the bulk of the work to compiled SQL queries.
This is a nice scenario because order books do not get very large (if we
want to process several order books, we can nicely partition the work by
order book across several machines), change extremely rapidly, and the
success of trading depends crucially on the speed with which the programmed
strategies process the data.

\item
Data warehouse loading: data integration and aggregation.
Commercial data warehouse loaders also use straight-line code for aggregation,
but the data integration queries that are executed before aggregation are currently not compiled. Compiling them only yields a really substantial improvement
if the integration and aggregation queries are compiled together. This may substantially reduce the time taken to load a warehouse. (Here we may have to scale up by using several machines.)
\end{itemize}

Claims / theses regarding query processing in main-memory databases (to be verified by experiments):
\begin{enumerate}
\item
When we compile queries to straight-line C code, it is best to use incremental
maintenance techniques to eagerly keep query results up to date.

\item
Index nested loop joins on main memory hash tables
dominate all other join techniques in the compiled
main-memory database setting.
It is worth having hash tables whose buckets are sorted, though (e.g. for
top-k query processing).

\item
There is something to be gained from bulk updates.
do we not just want to support bulk updates of one kind (e.g., just inserts),
but mixtures of inserts, updates, and deletes? Is this an optimization, and
worth the work?

\item
There is no point in adaptivity beyond bulk updates.
\end{enumerate}


Experiments:
Destroy classical databases (4 orders of magnitude improvement). Demonstrate
that incremental maintenance is a good idea for aggregates (two orders
of magnitude improvement over compiled non-incremental code).
}
