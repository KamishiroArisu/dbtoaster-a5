\section{Query Compilation}


\def\algsum{\mathrm{sum}}
\def\algagg{\mathrm{agg}}


\begin{figure*}
\begin{eqnarray}
\rho_{\vec{A}\vec{B}}(R) \bowtie \rho_{\vec{A}\vec{C}}(\{ \tuple{\vec{a}\vec{c}} \})
&\vdash&
\sigma_{\vec{A}=\vec{a}}(\rho_{\vec{A}\vec{B}}(R)) \times \rho_{\vec{C}}(\{ \vec{c} \})
\label{r1}
\\
\pi_{\vec{A}}(\rho_{\vec{A}}(R) \times \rho_{\vec{B}}(\{ \vec{b} \}))
&\vdash&
\rho_{\vec{A}}(R)
\label{r2}
\\
\rho_{A}(\{\vec{a}\}) \times \rho_{\vec{B}}(R)
&\vdash&
\sigma_{A=a}(\rho_{A}(\{\vec{a}\}) \times \rho_{\vec{B}}(R))
\label{r3}
\\
\algagg_{f(A,\vec{B})}(\sigma_{A=a}(R))
&\vdash&
\algagg_{f(a,\vec{B})}(\sigma_{A=a}(R))
\label{r4}
\\
\algsum_{f(\vec{A}) * g(\vec{B})}(\rho_{\vec{A}}(R) \times \rho_{\vec{B}}(S))
&\vdash&
\algsum_{f(\vec{A})}(\rho_{\vec{A}}(R)) *
\algsum_{f(\vec{B})}(\rho_{\vec{B}}(S))
\label{r5}
\\
\algsum_{a * f(\cdot)}(R)
&\vdash&
a * \algsum_{f(\cdot)}(R)
\label{r6}
\\
\algsum_{f(\cdot) + g(\cdot)}(R)
&\vdash&
\algsum_{f(\cdot)}(R) + \algsum_{g(\cdot)}(R)
\label{r7}
\\
\max_{a + f(\cdot)}(R)
&\vdash&
a + \max_{f(\cdot)}(R)
\label{r8}
\\
\max_{f(\vec{A}) + g(\vec{B})}(\rho_{\vec{A}}(R) \times \rho_{\vec{B}}(S))
&\vdash&
\max_{f(\vec{A})}(\rho_{\vec{A}}(R)) +
\max_{f(\vec{B})}(\rho_{\vec{B}}(S))
\label{r9}
\\
\max_{a * f(\vec{B})}(\rho_{\vec{B}}(R))
&\vdash&
\left\{
\begin{array}{lll}
a * \min_{f(\vec{B})}(\rho_{\vec{B}}(R)) & \dots & a < 0 \\
0                                        & \dots & a = 0 \\
a * \max_{f(\vec{B})}(\rho_{\vec{B}}(R)) & \dots & a > 0
\end{array}
\right.
\label{r10}
\\
\max_{f(\vec{A}) * g(\vec{B})}(\rho_{\vec{A}}(R) \times \rho_{\vec{B}}(S))
&\vdash&
\max\Big\{
\max_{f(\vec{A})}(\sigma_{f(\vec{A}) \ge 0}(\rho_{\vec{A}}(R))) *
\max_{g(\vec{B})}(\rho_{\vec{B}}(S)),
\nonumber\\
&& \quad\quad\;\;\,
\max_{f(\vec{A})}(\underbrace{\sigma_{f(\vec{A}) < 0}}_{\mathrm{optional}}(\rho_{\vec{A}}(R))) *
\max_{g(\vec{B})}(\sigma_{f(\vec{B}) \ge 0}(\rho_{\vec{B}}(S))),
\nonumber\\
&& \quad\quad\;\;\,
\min_{f(\vec{A})}(\sigma_{f(\vec{A}) < 0}(\rho_{\vec{A}}(R))) *
\min_{g(\vec{B})}(\sigma_{f(\vec{B}) < 0}(\rho_{\vec{B}}(S)))
\Big\}
\label{r11}
\\
\max_a(R)
&\vdash&
\left\{
\begin{array}{lll}
a       & \dots & \algsum_1(R) > 0 \\
-\infty & \dots & \algsum_1(R) = 0 \\
\end{array}
\right.
\label{r12}
\end{eqnarray}

In particular, if $g(\cdot) = 1$, then Rule~(5) gives
\[
\algsum_{f(\vec{A})}(\rho_{\vec{A}}(R)) *
\algsum_{1}(\rho_{\vec{B}}(S))
=
\algsum_{f(\vec{A})}(\rho_{\vec{A}}(R)) * |S|.
\]

\caption{Rewrite rules. agg can be either sum, max, or min.
count is sum$_1$.}
\label{fig:rules}
\end{figure*}



In this section, an expression of the form
\[
\algagg_f(\sigma_{\vec{A}=\vec{a}}(Q))[\vec{a}]
\]
is a map for an aggregate-group by queryd did use the environment, however in
that case the numbering for each line appeared on the following line (and not on
the same as I wanted to). That's why I decided that I will just have the two
equations without any te
\[
\mbox{select $\vec{A}$, agg($f$)
from $Q$
group by $\vec{A}$}.
\]
Given a group $\vec{a}$, the map returns
the aggregate value
$\algagg_f(\sigma_{\vec{A}=\vec{a}}(Q))$ for it.
An aggregate $\algagg$ (either sum, max, or min) returns exactly
one value -- the aggregate value.

The main rewrite step is the following.
We exploit the fact that for our aggregate functions $\algagg$,
\begin{multline*}
\algagg_f(\sigma_{\vec{A}=\vec{a}}(Q \cup \Delta Q))[\vec{a}]
= \\
\algagg(\algagg_f(\sigma_{\vec{A}=\vec{a}}(Q))[\vec{a}],
\algagg_f(\sigma_{\vec{A}=\vec{a}}(\Delta Q))[\vec{a}]).
\end{multline*}


Consider an aggregate-group by query
\[
\algagg_{f(\vec{a}, \vec{B}, \vec{C}, \vec{D})}
(\sigma_{\vec{A}=\vec{a}}(R_1 \bowtie \dots \bowtie R_k))[\vec{a}]
\]
where the schema of $R_1 \bowtie \dots \bowtie R_{k-1}$ is
$\vec{A}\vec{B}\vec{C}$ and the schema of
$R_k$ is $\vec{C}\vec{D}$.

W.l.o.g., we consider the case of an insertion of tuple
$\tuple{\vec{c},\vec{d}}$ into relation $R_k$.

We rewrite
$
\algagg_f(\sigma_{\vec{A}=\vec{a}}(\Delta Q))[\vec{a}]
$, that is,
\[
\algagg_{f(\vec{a}, \vec{B}, \vec{C}, \vec{D})}
(\sigma_{\vec{A}=\vec{a}}(R_1 \bowtie \dots \bowtie R_{k-1} \bowtie \{\tuple{\vec{c}\vec{d}}\}))[\vec{a}]
\]
to
\begin{equation}
\algagg_{f(\vec{a}, \vec{B}, \vec{c}, \vec{d})}
(\sigma_{\vec{A}\vec{C}=\vec{a}\vec{c}}(R_1 \bowtie \dots \bowtie R_{k-1}))[\vec{a}\vec{c}\vec{d}].
\label{eq:1}
\end{equation}

Figure~\ref{fig:rules} provides a set of rewrite rules, of which rules
\ref{r1}, \ref{r2}, \ref{r3}, and \ref{r4} are sufficient to  perform this
rewriting.

Since we would like the maps that have to be maintained to be as simple as
possible, we will try to express Equation~\ref{eq:1} in terms of 
aggregates
\begin{equation}
\algagg_{f'(\vec{a}, \vec{B}, \vec{c})}
(\sigma_{\vec{A}\vec{C}=\vec{a}\vec{c}}(R_1 \bowtie \dots \bowtie R_{k-1}))[\vec{a}\vec{c}]
\label{eq:2}
\end{equation}
which do not use $\vec{d}$.

In most cases
this is possible using the rewrite rules of Figure~\ref{fig:rules}.
Assuming that $f$ is an arithmetic expression built using addition and multiplication (and constants which may be negative), we can turn $f$ into an equivalent
expression that is a sum of products (by exploiting distributivity).

Let the aggregate be sum. In that case the rewriting of Equation~\ref{eq:1}
so as to eliminate the constants $\vec{d}$
is always possible using the rules \ref{r6} and \ref{r7}.
In the case of max, the rewriting is usually possible. The simplest case
where it is not is $f = B_1 * d + B_2$.


\begin{proposition}
To do: Formalize this fact.
\end{proposition}


Note that our rewriting has removed relation $R_k$ from the query to be
incrementally maintained. Inductively, we can solve the incremental maintenance
problem by just maintaining the maps constructed using this rewriting.


\begin{example}\em
Given schema $R(A,B)$, $S(B,C)$, $T(C,D)$.
We incrementally maintain the aggregate query
\[
s := \algsum_{A*D}(R \bowtie S \bowtie T).
\]
\begin{itemize}
\item
Insert R(a,b):
\begin{eqnarray*}
\Delta s &=& \algsum_{A*D}(\{\tuple{a,b}\} \bowtie S \bowtie T)
\\ &\stackrel{\ref{r1}}{=}&
\algsum_{A*D}(\{a\} \times \sigma_{B=b}(S) \bowtie T)
\\ &\stackrel{\ref{r3},\ref{r4}}{=}&
\algsum_{a*D}(\sigma_{B=b}(S) \bowtie T)
\\ &\stackrel{\ref{r6}}{=}&
a * \underbrace{\algsum_{D}(\sigma_{B=b}(S) \bowtie T)}_{s_D[b]}
\end{eqnarray*}

\item
Insert S(b,c):
\begin{eqnarray*}
\Delta s &=& \algsum_{A*D}(R \bowtie \{\tuple{b,c}\} \bowtie T)
\\ &\stackrel{\ref{r1}*}{=}&
\algsum_{A*D}(\sigma_{B=b}(R) \times \sigma_{C=c}(T))
\\ &\stackrel{\ref{r5}}{=}&
\underbrace{\algsum_{A}(\sigma_{B=b}(R))}_{s_A[b]} *
\underbrace{\algsum_{D}(\sigma_{C=c}(T))}_{s_D[c]}
\end{eqnarray*}

\item
Insert T(c,d): (analogous to insertion of R(a,b))
\begin{eqnarray*}
\Delta s &=& \algsum_{A*D}(R \bowtie S \bowtie \{\tuple{c,d}\})
\\ &=&
\underbrace{\algsum_{A}(R \bowtie \sigma_{C=c}(S))}_{s_A[c]} * d
\end{eqnarray*}
\end{itemize}

We incrementally maintain $s_D[b]$, $s_A[b]$, $s_D[c]$, and
$s_A[c]$ as well.

\begin{itemize}
\item
Insert R(a,b):
\begin{eqnarray*}
\Delta s_A[b] &=& \algsum_{A}(\{\tuple{a,b}\}) = a
\\
\mbox{foreach $c$: }
\Delta s_A[c] &=& \algsum_{A}(\{\tuple{a,b}\} \bowtie \sigma_{C=c}(S))
\\ &\stackrel{\ref{r1},\ref{r3},\ref{r4},\pi}{=}&
\algsum_{a}(\sigma_{BC=bc}(S))
\\ &\stackrel{\ref{r6}}{=}&
a * \underbrace{\algsum_{1}(\sigma_{BC=bc}(S))}_{s_1[b,c]}
\end{eqnarray*}

(Analogously insert T(c,d).)

\item
Insert S(b,c):
\begin{eqnarray*}
\Delta s_A[c] &=&
\algsum_{A}(R \bowtie \{\tuple{b,c}\})
\\ &\stackrel{\ref{r1}}{=}&
\algsum_{A}(\sigma_{B=b}(R) \times \{c\})
\\ &\stackrel{\pi}{=}&
\algsum_{A}(\sigma_{B=b}(R))
\;=:\; s_A[b]
\\
\Delta s_D[b] &=&
\algsum_{D}(\{\tuple{b,c}\} \bowtie T)
\\ &=&
\algsum_{D}(\{b\} \times \sigma_{C=c}(T))
\\ &=&
\algsum_{D}(\sigma_{C=c}(T))
\;=:\; s_D[c]
\end{eqnarray*}
\end{itemize}

Finally, we want to incrementally maintain $s_1[b,c]$:
\begin{itemize}
\item
Insert S(b,c):
\[
\Delta s_1[b,c] =
\algsum_{1}(\{\tuple{b,c}\}) = 1
\]
\end{itemize}

Thus the code is:
\begin{verbatim}
on insert into R values (a,b)
{
   s += a * s_D[b];
   s_A[b] += a;
   foreach c (in Cs[b]) do
      s_A[c] += a * s_1[b,c];
}

on insert into S values (b,c)
{
   s += s_A[b] * s_D[c];
   s_A[c] += s_A[b];
   s_D[b] += s_D[c];
   s_1[b,c] += 1;
}

on insert into T values (c,d)
{
   s += s_A[c] * d;
   s_D[c] += d;
   foreach b (in Bs[c]) do
      s_D[b] += s_1[b,c] * d;
}
\end{verbatim}
\punto
\end{example}


\begin{example}\em
Consider the same query as above where sum is replaced by max,
\[
m := \max_{A*D}(R \bowtie S \bowtie T).
\]

On insert into R values $(a,b)$: $m := \max(m, m')$ where
\begin{eqnarray*}
m' &=&
\max_{A*D}(\{\tuple{a,b}\} \bowtie S \bowtie T)
\\
&\stackrel{\ref{r1},\ref{r3},\ref{r4}}{=}&
\max_{a*D}(\sigma_{B=b}(S) \bowtie T)
\\
&\stackrel{\ref{r10}}{=}&
\left\{
\begin{array}{lll}
a * \min_{D}(\sigma_{B=b}(S) \bowtie T) & \dots & a < 0 \\
a * \max_{D}(\sigma_{B=b}(S) \bowtie T) & \dots & a \ge 0
\end{array}
\right.
\end{eqnarray*}

For the incremental maintenance of $S$ and $T$, we have to maintain
$\min_A[b] = \min_A(\sigma_{B=b}(R))$,
$\max_A[b] = \max_A(\sigma_{B=b}(R))$,
$\min_A[c] = \min_A(R \bowtie \sigma_{C=c}(S))$ and
$\max_A[c] = \max_A(R \bowtie \sigma_{C=c}(S))$.
For instance,
$\max_A[c] := \max(\max_A[c], \max_A'[c])$
where
\begin{eqnarray*}
\max_A[c] &=&
\max(\max_A[c], \max_A(\{\tuple{a,b}\} \bowtie \sigma_{C=c}(S)))
\\
&\stackrel{\ref{r1},\ref{r3},\ref{r4}}{=}&
\max(\max_A[c], \max_a(\sigma_{BC=bc}(S)))
\\
&\stackrel{\ref{r12}}{=}&
\left\{
\begin{array}{lll}
\max(\max_A[c], a) & \dots & s_1[b,c] > 0 \\
\max_A[c]          & \dots & s_1[b,c] = 0
\end{array}
\right.
\end{eqnarray*}
where we incrementally maintain
$s_1[b,c] = \algsum_1(\sigma_{BC=bc}(S))$.

Thus the code for maintaining the datastructures on an insert of $R(a,b)$ is
\begin{verbatim}
m = max(m, ((a<0) ? min_D[b] : max_D[b]));
max_A[b] = max(max_A[b], a);
min_A[b] = min(min_A[b], a);
if (s_1[b,c] > 0)
{
   max_A[c] =  max(max_A[c], a);
   min_A[c] =  min(min_A[c], a);
}
\end{verbatim}
\punto
\end{example}

\section{Alternate Compilation Rules}
Define a compiler $C = <Q, M, U>$, where $Q$ is a query, $M$ a set of map data
structures of the form ${attr}->aggregate(Q')$, and $U$ a set of update rules
for the maps, of the form $update: M -> unit$, that is these updates are
side-effects. We consider queries of the form:

\begin{center}
\texttt{select $A_{gb}$, aggregate(arithmetic($A_{agg}$)) from $G$}
\end{center}

where $A_{agg} = \{A_i\}$ are referred to as aggregation attributes, $A_{gb}$ as
grouping attributes, and $G=\{V,E\}$ denotes a join graph with relations
$V=\{R_i\}$ as nodes, and join predicates $E=\{\bigwedge_{i} a_i=b_i\}$ that are
conjunctive equality predicates.

\begin{figure*}
\begin{align}
<f(arith(A_{agg}), G), M, U> \wedge a \in A_{agg} \wedge distributive(f)
\vdash <f^*(f_{i}(arith(A_{agg}), G \cup A_i = a_i)), M, U>\\
\nonumber\\
<f(f_{i}(arith(A_{agg}), G \cup A_i = a_i)), M, U>
\wedge arith_1(A_i, arith_2(A_{agg} - \{A_i\}))
\wedge distributes(f_{i}, arith_1) \nonumber \\ 
\qquad \vdash <f(arith_1(a_i, f_{i}(arith_2(A_{agg} - \{a\}), G - R)),
\sigma_{A_i=a_i}(R) \Join_{E_R} G-R), M, U>\\
\nonumber\\
<f(arith_1(a, f_{i}(A_{agg}, G)), \sigma_{A=a}(R) \Join_{E_R} G), M, U>
\nonumber\\
\qquad \vdash <f_{i}(A_{agg}, G), M \cup \overrightarrow{B} \rightarrow
f_{i}(A_{agg}, G), U \cup {\lambda(\Delta R(a,\overrightarrow{b})). \Delta Q =
f(a . M_B[\overrightarrow{b}])>,
\mbox{ where } \overrightarrow{B} = Attrs(E_R)}\\
\nonumber\\
<f(arith(A_{agg}), G), M, U>
\wedge arith_1(A_i, arith_2(A_{agg} - \{A_i\})
\wedge associative(f_{i}, arith_1) \nonumber\\
\qquad \vdash <arith_1(f(A_i, \pi_{A_i}(G)),
f(arith_2(A_{agg} - \{A_i\}, \pi_{A_{agg} - \{A_i\}}(G)))), M, U>\\
\nonumber\\
<f(arith(A_{agg}), G), M, U>
\wedge distributive(f) \nonumber \\
\qquad \vdash <f^*(f_{i}(arith(A_{agg}),
G \cup S.\overrightarrow{B} = \overrightarrow{b_i})), M, U>, \mbox{ where }
S.\overrightarrow{B} = Attrs(E_S)\\
\nonumber\\
<f(f_{i}(arith(A_{agg}),
G \cup S.\overrightarrow{B} = \overrightarrow{b_i})), M, U>
\wedge arith_1(A_i, arith_2(A_{agg} - \{A_i\}))
\wedge distributes(f_{i}, arith_1) \nonumber \\
\qquad \vdash <f(arith_1(a_i, f_{i}(arith_2(A_{agg} - A_{i}), G - R),
\sigma_{A_i=a_i,\pi_R(B) = \pi_R(b_i)}(R) \times G - R \cup S.\overrightarrow{B}
= \overrightarrow{b}), M, U> \nonumber\\
\mbox{ where } S.\overrightarrow{B} = Attrs(E_S)\\
\nonumber\\
<f(arith_1(a_i, f_{i}(arith_2(A_{agg}), G),
\sigma_{A_i=a_i,\pi_R(B) = \pi_R(b_i)}(R) \times G \cup S.\overrightarrow{B}
= \overrightarrow{b}), M, U>
\nonumber\\
\qquad \vdash <f(arith(f_{i}(A_i,
R \cup \pi_R(\overrightarrow{B}) = \pi_R(\overrightarrow{b})),
f_{i}(arith(A_{agg}), G \cup S.\overrightarrow{B} = \overrightarrow{b})), M, U>
\end{align}
\end{figure*}



\section{Structural Query Decomposition}


The examples of the previous section have demonstrated how the rewrite
rules of Figure~\ref{fig:rules} can be used to decompose a complex
aggregate into smaller parts. For example,
we were able to decompose
\[
\Delta s = \algsum_{A*D}(R \bowtie S \bowtie T)
\]
on the insertion of $S(b,c)$ into
\[
\algsum_{A}(\sigma_{B=b}(R)) * \algsum_D(\sigma_{C=c}(T)).
\]

This decomposition was easy to find because the query was
very simple. For complex queries, we need some way of understanding the
join structure of the query to find good decomposition. The natural tool
for this are hypertree decompositions.


\begin{example}\em
The schema is
$R_1\{A,B,C\}$, $R_2\{B,D,E\}$, $R_3\{C,F,G\}$, $R_4\{F,H\}$,  $R_5\{G,I,J\}$,
and $R_6\{I,J,K,L\}$ and
the query is
\[
s := \algsum_{A*L}(\sigma_{HK=hk}(R_1 \bowtie \dots \bowtie R_6))[h,k].
\]
This query maliciously asks for tuples to be grouped by columns $H$ and $K$,
which are quite distant in the query's hypergraph. 

Let us insert tuple $R_1(a,b,c)$.
This is an acyclic query. This is a hypertree decomposition with $R_1$ as
the root node:
\comment{
\[
\psset{levelsep=12mm}
\pstree{\TR{\framebox{$R_1\{A,B,C\}$}}}
{
   \TR{\framebox{$R_2\{B,D,E\}$}}^B
   \pstree{\TR{\framebox{$R_3\{C,F,G\}$}}_{C; [H,K]}}
   {
      \TR{\framebox{$R_4\{F,H\}$}}^{F; [H]}
      \pstree{\TR{\framebox{$R_5\{G,I,J\}$}}_{G; [K]}}
      {
         \TR{\framebox{$R_6\{I,J,K,L\}$}}_{I,J; [K]}
      }
   }
}
\]
}

The edges are annotated with the columns that have to be passed between the nodes.
Now
\begin{eqnarray*}
\Delta s &=&
\algsum_{A*L}(\sigma_{HK=hk}(\{a,b,c\} \bowtie R_2 \bowtie \dots \bowtie R_6))[\cdot]
\\
&=&
a * \algsum_{L}(\sigma_{BCHK=bchk}(R_2 \bowtie \dots \bowtie R_6))[\cdot]
\\
&=&
a * \algsum_{L}(\sigma_{B=b}(R_2) \times \sigma_{CHK=chk}(R_3 \bowtie \dots \bowtie R_6))[\cdot]
\\
&=&
a * 
\algsum_1(\sigma_{B=b}(R_2))[b] \\
&*& 
\algsum_L(\sigma_{CHK=chk}(R_3 \bowtie R_4 \bowtie R_5 \bowtie R_6))[c,h,k].
\end{eqnarray*}

If we want to insert into $R_3$, we reroot the hypertree decomposition
\comment{
\[
\psset{levelsep=12mm}
\pstree{\TR{\framebox{$R_3\{C,F,G\}$}}}
{
   \pstree{\TR{\framebox{$R_1\{A,B,C\}$}}^C}
   {
      \TR{\framebox{$R_2\{B,D,E\}$}}^B
   }
   \TR{\framebox{$R_4\{F,H\}$}}^{F; [H]}
   \pstree{\TR{\framebox{$R_5\{G,I,J\}$}}_{G; [K]}}
   {
      \TR{\framebox{$R_6\{I,J,K,L\}$}}_{I,J; [K]}
   }
}
\]
}
and rewrite as follows:
\begin{eqnarray*}
\Delta s &=&
\algsum_{A*L}(\sigma_{HK=hk}(R_1 \bowtie R_2 \bowtie \{c,f,g\}
\\
&& \quad \bowtie R_4 \bowtie R_5 \bowtie R_6))[\cdot]
\\
&=&
\algsum_{A*L}(\sigma_{C=c}(R_1 \bowtie R_2) \times \sigma_{FH=fh}(R_4)
\\
&& \quad
\times \sigma_{GK=gk}(R_5 \bowtie R_6))[\cdot]
\\
&=&
\algsum_{A}(\sigma_{C=c}(R_1 \bowtie R_2))[c]
\\
&*& \algsum_{1}(\sigma_{FH=fh}(R_4))[fh]
\\
&*& \algsum_L(\sigma_{GK=gk}(R_5 \bowtie R_6))[gk]
\end{eqnarray*}
\punto
\end{example}


\begin{example}\em
If the query is acyclic, nothing really changes.
Consider the previous query where the schema of $R_5$ is now
$\{E,G,I,J\}$ (thus the query becomes cyclic).

The following is a hypertree decomposition rooted at $R_3$:
\comment{
\[
\psset{levelsep=12mm}
\pstree{\TR{\framebox{$R_3\{C,F,G\}$}}}
{
   \TR{\framebox{$R_4\{F,H\}$}}^{F; [H]}
   \pstree{\TR{\framebox{$R_1\{A,B,C\}, R_5\{E,G,I,J\}$}}_{C,G; [K]}}
   {
      \TR{\framebox{$R_2\{B,D,E\}$}}^{B,E}
      \TR{\framebox{$R_6\{I,J,K,L\}$}}_{I,J; [K]}
   }
}
\]
}
Thus $\Delta s$ for the insertion of $R_3(c,f,g)$ is
\begin{eqnarray*}
\Delta s &=&
\algsum_{A*L}(\sigma_{HK=hk}(R_1 \bowtie R_2 \bowtie \{c,f,g\}
\\
&& \quad \bowtie R_4 \bowtie R_5 \bowtie R_6))[\cdot]
\\
&=&
\algsum_{1}(\sigma_{FH=fh}(R_4))[fh]
\\
&*& 
\underbrace{\algsum_{A*L}(\sigma_{CGK=cgk}(R_1 \bowtie R_2 \bowtie R_5 \bowtie R_6))[cgk]}_{s_{A*L}[cgk]}
\end{eqnarray*}

On insertion of $R_5(e,g,i,j)$,
\begin{eqnarray*}
\Delta s_{A*L}[cgk]
&=&
\algsum_{A*L}(\sigma_{CGK=cgk}(R_1 \bowtie R_2 \\
&& \quad \bowtie \{\tuple{e,g,i,j}\} \bowtie R_6))[\cdot]
\\
&=&
\algsum_{A*L}(\sigma_{C=c}(R_1) \bowtie \sigma_{E=e}(R_2) \\
&& \quad \times \sigma_{IJK=ijk}(R_6))[\cdot]
\\
&=&
\algsum_{A}(\sigma_{C=c}(R_1) \bowtie \sigma_{E=e}(R_2))[ce] \\
&& \quad * \algsum_L(\sigma_{IJK=ijk}(R_6))[ijk]
\end{eqnarray*}
\punto
\end{example}
