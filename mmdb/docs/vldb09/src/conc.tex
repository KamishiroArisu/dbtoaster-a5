\section{Conclusions and Future Work}

We have presented \compiler, a SQL compiler that comprehensively shows the
benefits of creating compiled query executors for main-memory procesing of
repetitive and standing queries. Our executor is based on the notion of
processing individual tuples, by combining them with aggregated views defined
over the remaining parts of the query. The tuple function is also responsible for
maintaining these aggregate views, and this turns out to be extremely simple and
cheap in the main memory context given our novel map algebra which can readily
convert single tuple deltas into simple arithmetic expressions of parameterized
queries. Compiled query executors generated by \compiler\ dominate state of the
art database systems, including relational databases and stream processing on a
variety of application scenarios, and is only rivalled in a small set of use
cases by a compiled query executor implementing a plan in its original
operator-centric form. In our view, this raises the possibility that such an
approach should be the de facto processing methodology, as opposed to the
plan-based approach in future query executors.

We view the topic of query compilation and compiled executors as still in its
infancy, and hope that \compiler\ is able to raise interest in this domain. 
As part of ongoing and future work, we plan to investigate the efficiency of
our generated code and understand its memory behavior. A large part of this
includes extending to other non-relational operations in SQL, such as top-k,
and understanding the data structures needed to process tuple functions for
such operations. Furthermore, we would like to apply tuple function
compilation and execution in other novel domains where the database
community is branching out, including scientific databases for executing
simulations, as well as for probabilistic inference on graphical models, where
the common theme in both these domains is a graphical representation of 
computation. Finally the question of getting the right balance of adaptivity
and compilation, especially in light of the difficulty in managing large-scale
datbaase systems, remains an open question. We would like to integrate our
compilation techniques with the emerging techniques for providing self-managing
databases to address this issue,

\comment{
\begin{itemize}
\item
Programming language theory paper: how to compile SQL to use user-specified
data structures.

\item
Compiling down more general declarative languages, with operators and
aggregates whose semantics is definable in the framework. Applications e.g.\
probabilistic inference in graphical models. Compile the graph structure
of the graphical model and the inference algorithms (using e.g.\ mesage
passing techniques) into C code. The data to be processed are the factor
relations/conditional probability tables.

\item
Compiling database applications and the database server together into a single
program that runs robustly and requires no administration. I.e., combine
our query compilation work with automatic administration ideas, if any of
this is actually needed.
\end{itemize}
}