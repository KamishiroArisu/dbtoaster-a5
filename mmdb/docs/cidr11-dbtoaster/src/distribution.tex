Scaling up a DDMS requires not only storing progressively more data, but also a dramatic increase in computing resources.  As alluded to in Section \ref{sec:storage}, DDMS are amenable to having their data distributed across a cluster.  We have implemented a cluster-based execution runtime as part of DBToaster.

\comment{Unlike ad-hoc query systems [MapReduce,HDFS], a distributed DDMS can use advance knowledge of the data usage patterns to better coordinate the distribution of processing with its data placement scheme}

DBToaster considers two general techniques for distributing data and computation: (1) Compute the data where it will be stored, or (2) Store the data where it will be used.  The latter approach necessitates potentially extensive replication of data across the cluster, but enables the use of optimistic computation techniques that substantially reduce computational latency.

\medspace 

{\bf Creating a Total Order}\/.
The total ordering of transitions between database steps provides DBToaster with a clean synchronization abstraction; Each transition is conditioned on the prior state of the database and potential inter-transition conflicts are known at compile time.  Unfortunately, imposing such a total ordering in a distributed setting is not straightforward.  Clients trying to trigger updates in parallel will easily overwhelm a centralized sorting node, at least at the scales we are interested in.  

Instead, DBToaster imposes a logical total ordering by combining rough clock synchronization with an arbitrary total ordering over the clients.  The total ordering ensures that a transition sees only the effects of prior transitions - though it may not see all of them.  This forces DBToaster to support out-of-order updates, which it does by optimistically applying the transition and recomputing those portions later changed by transitions occurring earlier in the logical ordering.

\medspace

{\bf Hybrid Consistency}\/.
DBToaster's out-of-order mechanism implements an eventual consistency model that may not be sufficient for all data processing tasks.  DBToaster uses a background task to periodically identify the furthest point in the update stream that has been fully propagated, and produces a consistent snapshot from the data at that point.  Interestingly, this background task is already necessary for garbage collection purposes.  Thus, the same system can simultaneously produce a low-latency eventual consistency view of the database as well as a periodic consistent snapshot of the same data, an interface similar that provided by\cite{bayou}.

\medspace

{\bf Availability}\/.
One other interesting point in the distributed design space is that of availability and recovery.  
%Traditional stream processors dwell entirely in-memory for efficiency reasons, and rely on replication-based strategies to provide high availability.  Conversely, t
The large state found in a DDMS necessitates the use of out-of-core storage.  Instead of over-provisioning replicas and consuming precious network bandwidth maintaining them, DBToaster can ensure recoverability by maintaining data in a persistent store.  Tuning the tradeoffs between these various options, however, remains an open issue.

