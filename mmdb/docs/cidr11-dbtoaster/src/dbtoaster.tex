We present an example of designing a transition function between database
states, in this case focusing on dynamic data driving view maintenance of
standing queries comprising the visible schema. The underlying conceptual model
of a database management system as a state machine has driven our prior work on
this topic in the DBToaster project [REFS], and here we discuss how the concept
of transitioning entire database states has led to a novel view maintenance
algorithm. Furthermore the need to apply this transition with high frequency
motivates precomputing and compiling the transition function into extremely
efficient code.

This section is intended to convey that our database-as-state model can lead to
significant rethinking of existing methods throughout a data management system,
and novelty, algorithmically and architecturally, in designing a system to
handle dynamic data. Throughout this section, we use the term transition to
refer to both the update itself, and the work required to evolve the database
following the application of the update to base tables.

\vspace{1mm}
\subsection{View Maintenance in DBToaster}
Given a database state, existing view maintenance techniques will incrementally
handle transitions to another state for a given update. We can informally
represent this with a triple $\tuple{q,m,q'}$, corresponding to the view query,
the materialization of that query, and the delta query responsible for
maintaining the materialization.
\note{Should talk more about delta queries if you're going to make this the
first mention of it, i.e. with an example}
On an update, view maintenance performs the work: $m_{new} = m_{old} + q'(u)$,
and this is guaranteed to ensure $m_{new} = q(db_{new})$. That is, a
materialized view can be updated with the result of a delta query taking an
update $u$ as an argument, and the view maintenance technique must ensure the
updated view is equivalent to the query result on the modified database. Here,
the delta query can be thought of as a parameterized SQL query, with parameters
corresponding to attributes in the update.

With our conceptual model, we are able to make the
following key insight when taking a holistic approach with the state machine
model, namely that repeatedly applying the same transition with current
techniques results in significantly redundant work. While a transition results
in incremental processing in terms of the part of the database affected by the
update, it is not incremental with respect to the remainder of the database. The
transition evalutes delta queries from scratch on the remainder of the database,
rather than leveraging the fact that this remnant has not changed, and any work
done previously on that portion of the database can be reused.
\note{Diagram here:
\comment{states containing base relations and a query,
transitions for all base relations, each leading to another state. For one of these
neighboring states, we'll repeatedly apply the same transition, highlighting
that the remainder of the base tables do not change, yet the delta queries are
still evaluated from scratch.}
}

To facilitate reuse, we materialize the delta query over the remainder tables,
making it part of the auxiliary state of the database. We refer to this as the
view state, and it is used in our view maintenance approach. Subsequently our
transitions must maintain the view state, leading to the concept of higher-order
delta queries. Higher-order deltas are determined through a recursive processing
of materializing a delta query, and then incrementally maintaining the
materialized result (which would involve further materialization, and delta
queries and so on). That is we can define further triples,
$\tuple{q', m', q''}$, corresponding to the delta query, its materilization,
and a second-order delta query, and so on with $\tuple{q'',m'',q'''}$.
\note{Diagram here.}

This process does not continue forever and terminates given one important
property of computing delta queries: \textit{a delta query is often simpler than
its parent query}. In particular \textit{k}-th order delta query, $q^k$ has
fewer input relations than a \textit{k-1} order query $q^{k-1}$, but additional
parameters corresponding to attributes that are present in $q^{k-1}$. This
provides an informal overview of our view maintenance approach, and we present
an algorithm below to compute both the view state being materialized and the
higher-order delta queries that maintains the view state. The algorithm yields a
\textit{transition program}, essentially trigger function that efficiently
executes a transition from one database to another, including both the visible
and auxiliary state. For the view maintenance problem, the transition program is
simply a sequences of updates to materialized views by delta queries, each
update being of a different order.

\vspace{1mm}
\subsection{Compiling Queries to Transition Programs}
\noindent To present our algorithm, we first describe our query representation,
tailored for incremental processing, and a simple and powerful set of transformations
that we use to simplify and optimize queries.

\tinysection{Query Language}
\noindent Our query language is described by the following EBNF:

\def \calcsum{\mbox{Sum}}
\def\calceq{\mbox{{\tt =}}}
\def\calcgt{\mbox{{\tt >}}}
\def\calcgte{\mbox{{\tt >=}}}
\def\calclte{\mbox{{\tt <=}}}
\def\calclt{\mbox{{\tt <}}}

\def \q{q}
\def \qa{q_1}
\def \qb{q_2}
\def \v#1{\mbox{#1}}
\def \vv#1{\mbox{{\tiny #1}}}
\def \z{\mathbb{Z}}

\vspace{-3mm}
\begin{align*} 
\q \; \mbox{::-} &
  \;    c \;|\; x
  \;|\; R(\vec{x}) \;|\; m(\vec{x},\vec{y})
\\
| & \; \calcsum(\vec{x}, \q)
  \;|\; \q + \q \;|\; \q * \q  \;|\; -\q
  \;|\; \q \; \theta \; 0 \;|\; x := \q
\end{align*}

The grammar represents basic terms of queries such as constants, variables, and
relations $R$ (with attributes $\vec{x}$). We represent materialized views, $m$,
and refer to them as \textit{maps} since these are the underlying in-memory data
structures. Materialized views, or maps, are parameterized SQL queries that
accept parameters $\vec{x}$, and yield schema $\vec{y}$.
Specifically, a map is a \textit{materialized} parameterized query,
where we materialize for all parameter values seen during continuous query
execution (the \textit{active domain} of the parameter).
The intuition behind the representation of views as parameterized SQL comes from
our need to materialize delta queries, which have parameters according to the
update they handle.

For complex terms, the grammar captures sum aggregates
$Sum(\vec{x},q)$ with group-by attributes $\vec{x}$, three generalized
arithmetic operators, addition, multiplication and additive inverse, a
comparison operator ($\theta \in \{=,\neq,<,\leq,>,\geq\}$), and variable
assignment ($:=$). For readability, we write $\calcsum(\q)$ if an aggregate
has no group-bys, and simplify the syntax of comparison as $\theta(\q)$ since
the RHS is 0. 
Our language is quite rich, for example it can represent nested scalar
queries such as the VWAP query in Section [REF]:

\comment{
\begin{verbatim}
select sum(price * vol) from bids b0
where 0.25 * (select sum(b1.vol) from bids b1) >
     (select sum(b2.vol) from bids b2
      where b2.price > b0.price)
\end{verbatim}
}

\vspace{-3mm}
\begin{align*}
\calcsum(
& B(\v{P0,V0}) * \v{P0} * \v{V0} *\\
& \calcgt(0.25 * \calcsum(B(\v{P1,V1}) * \v{V1}) \\
& \qquad \; \; - \calcsum(B(\v{P2,V2}) * \v{V2} * \calcgt(\v{P2-P0}))))
\end{align*}

We briefly discuss expression semantics, starting with maps. Strictly speaking,
\textit{all of the expressions in our grammar (i.e. queries) represent maps},
and maps associate keys to values. Recall our analogy between maps
$m(\vec{x},\vec{y})$ and parameterized SQL queries. Map $m$ has a key
$\tuple{\vec{x},\vec{y}}$, namely the parameters (bound variables, which must be
provided when evaluating a parameterized query) and schema attributes (free
variables, returned by the query). For readability, we use a doubly-indexed map
notation, $m[\vec{x}][\vec{y}]$, instead of writing in pair notation throughout
$m[\tuple{\vec{x}, \vec{y}}]$. We define map values with a generalized multiset
relation [PODS REF] model of a database, where a relation $R$ is a map whose
value yields the cardinality of that tuple in a multiset.

We assume that the user queries provided as input to compilation contain no
maps, only relations and other terms as is standard in SQL. Our compilation
algorithm rewrites queries to include maps. Due to space constraints, we focus
on two arithmetic operations, addition and multiplication. First, we present an
example of addition:

\vspace{-7mm}
\begin{figure}[h]
\begin{minipage}{1in}
\[
\begin{array}{ll|ll|l}
\multicolumn{5}{c}{m1}\\
 a & c & b & d & \\
\hline
 1 & 1 & 2 & 1 & 3 \\
\comment{
 1 & 2 & 4 & 2 & 1 \\
 1 & 2 & 3 & 3 & 4 \\
}
 1 & 2 &   & 4 & 2
\end{array}
\]
\end{minipage}
+
\begin{minipage}{0.85in}
\[
\begin{array}{ll|l|l}
\multicolumn{4}{c}{m2}\\
 a & c & d & \\
\hline
 1 & 1 & 1 & 1 \\
 1 & 2 & 4 & 2 \\
\end{array}
\]
\end{minipage}
=
\begin{minipage}{1in}
\[
\begin{array}{ll|ll|l}
\multicolumn{5}{c}{m1 + m2}\\
 a & c & b & d & \\
\hline
 1 & 1 & 2 & 1 & 3 \\
 1 & 1 &   & 1 & 1 \\
\comment{
 1 & 2 & 4 & 2 & 1 \\
 1 & 2 & 3 & 3 & 4 \\
}
 1 & 2 &   & 4 & 4
\end{array}
\]
\end{minipage}
\end{figure}

\vspace{-3mm}
\noindent Above, we have two maps $m1, m2$, where blanks in columns represent
\textit{holes}. Holes are not the same as nulls, they determine when attributes
can be ignored for consistency across tuples.
\note{Explain consistency.}
Addition is a schemaless union where the result contains the union of
operand maps' parameters and result schemas, and adds the values of map entries.

Next, we discuss multiplication. Multiplication is the key
connective in our language since it performs binding propagation, similarly to
datalog, where one map's result schema contains attributes used for
another map's parameters, as seen in $m1[a][b1,b2] * m2[b2,c][d]$.
Here, $b2$ is propagated from map $m1$ to $m2$. The result of this
operation ensures consistency of $b2$ between maps $m1, m2$ (inconsistent
tuples are not included in the output), for example:

\vspace{-4mm}
\begin{figure}[h]
\begin{minipage}{1in}
\[
\begin{array}{l|ll|l}
\multicolumn{4}{c}{m1}\\
 a & b_1 & b_2 & \\
\hline
 1 & 2 & 4 & 3 \\
 2 & 4 & 1 & 1 \\
 2 & 3 & 2 & 4
\end{array}
\]
\end{minipage}
*
\begin{minipage}{0.85in}
\[
\begin{array}{ll|l|l}
\multicolumn{4}{c}{m2}\\
 b_2 & c & d & \\
\hline
 1 & 1 & 1 & 1 \\
 2 & 2 & 4 & 2 \\
\end{array}
\]
\end{minipage}
=
\begin{minipage}{1in}
\[
\begin{array}{ll|ll|l}
\multicolumn{5}{c}{m1 * m2}\\
 a & c & b_1 & d & \\
\hline
 2 & 1 & 4 & 1 & 1 \\
 2 & 2 & 3 & 4 & 8 \\
\end{array}
\]
\end{minipage}
\end{figure}

\vspace{-3mm}
\noindent Above, $b2$ is excluded from the result key since it is a propagated
attribute. The tuple $\tuple{a=1,b1=2,b2=4 \mapsto 3}$ in $m1$ does not
contribute to the result since there is no consistent value of $b2$ in $m2$.
Multiplication is a generalized equi-join, and in the absence of any
propagation, a generalized Cartesian product that operates on both levels of
maps.
\todo{Explain variables, i.e. we don't materialize attributes but these are
bound from the left.}
Binding propagation is critical for a few reasons: i) for
multiplication with variables, it allows a map's values to be combined with the
variable's value; ii) for multiplication with constraints, it yields maps
with 0-1 values indicating whether entries pass the constraint; iii) for
multiplication with other maps, where it facilitiates equi-joins and correlated
attributes in nested queries.


\comment{
We write these semantics as
$R := \tuple{} \mapsto \vec{x} \mapsto R(\vec{x})$, indicating the map has a key
with no parameters, schema $\vec{x}$, and value given by the cardinality $R(x)$
of the multiset $R$. The \textit{type} of the map is
$R : \tuple{} \mapsto \vec{x} \mapsto \z$, thus the semantics indicates the
result key, and the map value.

The semantics of constants and variables are: $c := \tuple{} \mapsto
\tuple{} \mapsto c$ and $Var(x) := \vec{b} \mapsto \tuple{} \mapsto \vec{b}(x)$.
The latter states that variables get their value from query parameters.
Due to space constraints, we briefly present the semantics of complex terms.


\vspace{-4mm}
\begin{align*}
\calcsum(\vec{y}, \q[\vec{x}][\vec{y}\vec{z}]) := & \;
\vec{x} \mapsto \vec{y} \mapsto \sum_{\vec{z}} \q
\\
\qa[\vec{u}][\vec{v}] + \qb[\vec{w}][\vec{x}] := & \;
\vec{y} \mapsto \vec{z} \mapsto
\qa[\vec{y}][\vec{z}] + \qb[\vec{y}][\vec{z}]
\\
& \mbox{ where $\vec{y} = \vec{u} \cup \vec{w}, \vec{z} = \vec{v} \cup \vec{x}$}
\\
-\q[\vec{x}][\vec{y}] := & \; \vec{x} \mapsto \vec{y} \mapsto -(\q)
\\
\q[\vec{x}][] \; \theta \; 0 := & \; \vec{x} \mapsto \tuple{} \mapsto
                    \begin{cases}
                    1 \ldots \q \; \theta \; 0\\
                    0 \ldots \mbox{otherwise.}
                    \end{cases}
\\
(x := \q[\vec{b}][]) := & \; \vec{b} \mapsto \tuple{x=q} \mapsto 1
\end{align*}

Above, we write typing constraints of subexpressions on the left-hand
side of semantics definitions. Thus a sum aggregate with group-by $\vec{y}$
over a map $q[\vec{x}][\vec{yz}]$ adds up values of $\vec{z}$. 
\note{Talk about consistency.}
The addition operator ($+$) is a schemaless union over both parameters and
result schema of its inputs. Additive inverse simply negates the map value,
preserving its type. Predicates yield maps with 0-1 values, and are constrained
to have no output schema. They are singletons, just like constants and
variables, and in this way, our language supports nested scalar queries such as
nested sum aggregates. Finally variable assignment $x := q$ yields a singleton
map with schema $x$, and key (of value) $q$. This leaves the multiplication
operator, which in our langauge is capable of propagating parameters, much like
datalog. Suppose we have: $\qa[\vec{u}][\vec{v}] * \qb[\vec{w}][\vec{x}] :=$

\vspace{-4mm}
\begin{align*}
& \vec{y} \mapsto
\sum_{\{y\} = \{u\} \Join \{w\}} 
\vec{z} \mapsto
  \sum_{\{z\} = \{v\} \Join \{x\}}
  \qa[\vec{y}][\vec{z}] * \qb[\vec{y}][\vec{z}]
\\
& \ldots \mbox{when $\vec{v} \cap \vec{w} = \emptyset$}
\\
& \vec{y} \mapsto
\sum_{\{y\} = \{u\} \Join \{b\}} 
\vec{z} \mapsto
  \sum_{\{z\} = \{v\} \Join \{ax\}}
  \qa[\vec{y}][\vec{z}] * \qb[\vec{y}][\vec{z}]
\\
& \ldots \mbox{otherwise, where $\vec{v} \cap \vec{w} = \vec{a}, 
\vec{w} - \vec{v} = \vec{b}$}
\end{align*}

Above, the first case describes when no propagation occurs. The result map has
parameters and schema according to the join of the operands' keys and value as a
product of operand values. In the second case we have propagation of the LHS
query's result schema being bound as parameters in the RHS query. Thus the
result map does not consider propagated attributes as parameters itself.
}


\vspace{0.5mm}
\tinysection{Query Transformations}
We present a set of query transformations, through examples for brevity, that
can be used by our algorithm to simplify and optimize queries. The first is a
canonicalization of queries into a \textit{recursively polynomial} form. A
polynomial query is a query represented as a sum of \textit{monomials}, where
monomials are products of any other type of query (constraints, aggregates, etc.
but not sums or products). This is a polynomial:
\begin{align*}
\calcsum( & R(\v{A,B}) * S(\v{C,D}) * \calcgt(\v{B}-\v{C}) *
\calceq(\v{A}-\v{D})) * \\
& \calcsum(T(\v{E,F}) * \v{E}) + -\calcsum(U(\v{G,H}) * \calceq(\v{H}-5))
\end{align*}
The following query is not a polynomial:

$\calcsum(R(\v{A,B}) + (S(\v{C,D}) * \v{C})) *
 \calcsum(T(\v{E,F}) * \calceq(\v{E}-10))$

\noindent because of the addition inside the first sum aggregate. Our normal
form is recursively defined due to the existence of nested queries. That is each
nested query must also be in polynomial form. We require this normal form to
easily apply the transformations described next.

To simplify queries, we apply unification, which facilitates variable
elimination, and then factorization, which separates a complex monomial into a
product of simpler monomials. First we give an example of unification and
variable elimination:

Now, we demonstrate factorization:

Above, \ldots

Finally, we present a transformation related to binding propagation, called
preaggregation. This ensures that our product operation propagates the minimal
number of variables, when considering a left-associative multiplication.


\tinysection{Incremental Processing}
So far, we have described queries, with no mention of incremental processing. We
now describe a transformation to compute the \textit{delta} of a query, which
computes the increment for a query given an update to one of its input
relations. The \textit{delta is itself a query}, and conceptually
is derived by replacing the input relation being updated with a set of
parameters. These parameters are passed to the transition program, thus the
delta query can be thought of as a parameterized SQL query. Our transformation
rules for computing delta queries are:

\vspace{-3mm}
\begin{align*}
\comment{
\Delta_{+R(\vec{x} \mapsto \vec{t})} c := & \; 0
\\
\Delta_{+R(\vec{x} \mapsto \vec{t})} Var(y) := & \; 0
\\
}
\Delta_{+R(\vec{x} \mapsto \vec{t})} R(\vec{x}) := & \;
\prod_i^{sch(\vec{x})} x_i := t_i
\\
\comment{
\Delta_{+R(\vec{x} \mapsto \vec{t})} S(\vec{y}) := & \; 0
\\
}
\Delta_{+R(\vec{x} \mapsto \vec{t})}
\calcsum(\vec{x},\q) := & \; \calcsum(\vec{x},\Delta\q)
\\
\Delta_{+R(\vec{x} \mapsto \vec{t})} (\qa + \qb) := & \;
(\Delta\qa + \Delta\qb)
\\
\Delta_{+R(\vec{x} \mapsto \vec{t})} (\qa * \qb) := & \;
(\Delta \qa * \qb) +
(\qa * \Delta \qb) +
(\Delta \qa * \Delta \qb)
\\
\Delta_{+R(\vec{x} \mapsto \vec{t})} \theta(\q) := & \;
(\theta(\q + \Delta \q) * \overline{\theta}(\q)) -
(\overline{\theta}(\q+\Delta \q) * \theta(\q))
\\
\Delta_{+R(\vec{x} \mapsto \vec{t})} -\q := & \;
    -(\Delta \q)
\\
\Delta_{+R(\vec{x} \mapsto \vec{t})} (x := \q) := & \;
    (x := \Delta \q)
\end{align*}

The above top-down rules specify deltas for complex terms. In addition, deltas
of constants and variables yield 0, and the delta of a relation other than the
one being updated also yields 0 (i.e. the rule: $\Delta_{+R(\vec{x} \mapsto
\vec{t})} S(\vec{y}) := \; 0$). Also note, we have no rule for maps, since maps
do not appear in user queries. All of these transformations are applied before
our algorithm introduces maps. These rules are similar to those found in
incremental view maintenance, with the exception of our rule for constraints,
which can handle nested queries. To the best of our knowledge, we have not seen
this addressed before. We wrap up this transformation with two insights: i)
delta queries are nearly always \textit{simpler} than their parent queries, with
the exception of nested queries; ii) queries are closed under taking deltas,
that is our delta queries are in the same language fragment, they too are SQL
queries.
We will need to build a main-memory SQL engine, and we talk about novel
approaches with code generation in our algorithm.

\tinysection{Compiling Transition Programs}
We can now piece together the various transformations above, to perform
\textit{recursive query compilation}. Our algorithm produces transition
programs, which have the following grammar:

\begin{itemize}
  \item Recall that queries are closed under taking deltas. Thus the RHS of our
  map updates are SQL queries, which we'll represent in our form as described
  above.
  \item Initial values
  \item Simplified compilation algorithm
    \begin{itemize}
    \item Additional steps beyond sequentially applying transformations
    above:
    \item loop var simplification
    \item initializer expression computation
    \end{itemize}
  \item Code generation: structural recursion to generate main-memory engines
    \begin{itemize}
      \item \note{Can't go into details of full functional language
      representation}
      \item Convey high-level goals of functional language representation:
        \begin{itemize}
        \item Function composition and inlining
        \end{itemize} 
      \item Give two simplified examples of programs, one showing join tree,
      one showing multiway join.
      \item Mention tradeoffs in two programs.
      \item Mention no known programmatic representation of physical level
      details of operators. We can do this because we're reasoning about data
      structures, not operators!
      \item Give listing of physical properties we can play with:
      \begin{itemize}
        \item production of intermediate results
        \item pipelining
        \item 
      \end{itemize}
    \end{itemize}
\end{itemize}