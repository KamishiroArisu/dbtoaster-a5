We present an example of designing a transition function between database
states, in this case focusing on dynamic data driving view maintenance of
standing queries comprising the visible schema. The underlying conceptual model
of a database management system as a state machine has driven our prior work on
this topic in the DBToaster project [REFS], and here we discuss how the concept
of transitioning entire database states has led to a novel view maintenance
algorithm. Furthermore the need to apply this transition with high frequency
motivates precomputing and compiling the transition function into extremely
efficient code.

This section is intended to convey that our database-as-state model can lead to
significant rethinking of existing methods throughout a data management system,
and novelty, algorithmically and architecturally, in designing a system to
handle dynamic data. Throughout this section, we use the term transition to
refer to both the update itself, and the work required to evolve the database
following the application of the update to base tables.

\vspace{1mm}
\subsection{View Maintenance in DBToaster}
Given a database state, existing view maintenance techniques will incrementally
handle transitions to another state for a given update. We can informally
represent this with a triple $\tuple{q,m,q'}$, corresponding to the view query,
the materialization of that query, and the delta query responsible for
maintaining the materialization. On an update, view maintenance performs the
work: $m_{new} = m_{old} + q'(u)$, and this is guaranteed to
ensure $m_{new} = q(db_{new})$. That is, a materialized view can be updated with
the result of a delta query taking an update $u$ as an argument, and the view
maintenance technique must ensure the updated view is equivalent to the query
result on the modified database. Here, the delta query can be thought of as a
parameterized SQL query, with parameters corresponding to attributes in the
update.

With our conceptual model, we are able to make the
following key insight when taking a holistic approach with the state machine
model, namely that repeatedly applying the same transition with current
techniques results in significantly redundant work. While a transition results
in incremental processing in terms of the part of the database affected by the
update, it is not incremental with respect to the remainder of the database. The
transition evalutes delta queries from scratch on the remainder of the database,
rather than leveraging the fact that this remnant has not changed, and any work
done previously on that portion of the database can be reused.
\note{Diagram here: states containing base relations and a query, transitions
for all base relations, each leading to another state. For one of these
neighboring states, we'll repeatedly apply the same transition, highlighting
that the remainder of the base tables do not change, yet the delta queries are
still evaluated from scratch.}

To facilitate reuse, we materialize the delta query over the remainder tables,
making it part of the auxiliary state of the database. We refer to this as the
view state, and it is used in our view maintenance approach. Subsequently our
transitions must maintain the view state, leading to the concept of higher-order
delta queries. Higher-order deltas are determined through a recursive processing
of materializing a delta query, and then incrementally maintaining the
materialized result (which would involve further materialization, and delta
queries and so on). That is we can define further triples,
$\tuple{q', m', q''}$, corresponding to the delta query, its materilization,
and a second-order delta query, and so on with $\tuple{q'',m'',q'''}$.
\note{Diagram here.}

This process does not continue forever and terminates given one important
property of computing delta queries: \textit{a delta query is often simpler than
its parent query}. In particular \textit{k}-th order delta query, $q^k$ has
fewer input relations than a \textit{k-1} order query $q^{k-1}$, but additional
parameters corresponding to attributes that are present in $q^{k-1}$. This
provides an informal overview of our view maintenance approach, and we present
an algorithm below to compute both the view state being materialized and the
higher-order delta queries that maintains the view state. The algorithm yields a
\textit{transition program}, essentially trigger function that efficiently
executes a transition from one database to another, including both the visible
and auxiliary state. For the view maintenance problem, the transition program is
simply a sequences of updates to materialized views by delta queries, each
update being of a different order.

\vspace{1mm}
\subsection{Query Compilation and Transition Programs}
To present our algorithm, we first describe our query representation, tailored
for incremental processing, and a simple and powerful set of transformations
that we use to simplify and optimize queries.

\tinysection{Query Language}
\begin{itemize}
  \item Query language
  \item SQL to our query language
  \item Transformations and optimizations in our language
\end{itemize}

\tinysection{Incremental Processing}
\begin{itemize}
  \item Incremental processing
  \item Deltas, deltas as parameterized queries
\end{itemize}

\tinysection{Compiling Transition Programs}
\begin{itemize}
  \item 
\end{itemize}