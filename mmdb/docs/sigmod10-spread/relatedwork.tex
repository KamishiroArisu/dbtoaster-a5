\nop{

\section{Related Work}
\label{sec:relatedwork}


Database compilers \cite{DBLP:conf/pods/Batory88,DBLP:journals/jiis/BatoryT97}.



A large body of previous research has focused on incremental view maintenance
of relational database queries
\cite{roussopoulos-tods:91,griffin-sigmod:95,colby-sigmod:96,yan-vldb:95,kotidis-tods:01,zhou-vldb:07}.
The focus of this work was always on only one level of delta rewriting
and then using classical relational query processing techniques
based on interpreted query plans and heavyweight monolithic query operators
(such as joins).
In general, however, this means
that the rewritten query is still an arbitrarily complex query (in the number
of joins) and the message passing technique employed in Cumulus is not
applicable.

Most of the previous work on incremental view maintenance does not observe or leverage the fact that aggregation in queries can actually make incremental maintenance {\em easier}, rather than harder, because it greatly reduces the amount of data in the result and because numbers have additional algebraic properties that relations do not have and which allow for further decomposition and optimization of queries.
}

\nop{
Column stores \cite{DBLP:journals/tods/Batory79,DBLP:conf/cidr/KerstenM05,DBLP:conf/vldb/StonebrakerMAHHH07}
are a prominent and successful instance of the recent trend
to abandon rather old database technologies and try out new architectural
para\-digms. Column stores have also been shown to be particularly well suited
for OLAP. However, column stores are fundamentally a secondary storage
concept; in main memory, it does not seem to be particularly meaningful to talk of column or row stores. To date, there has also been no work
on incremental view maintenance in column stores.
Indeed, they are notoriously bad at updating (worse than classical row stores).

H-Store is an OLTP system that abandons classical concurrency control and
recovery baggage for very high performance. It is interesting for the
fresh look at things, and for the viewpoint that even OLTP systems should
be much simpler, nimbler systems than classical databases, which is also
part of the vision behind Cumulus. However, H-Store solves a problem very
different from the one addressed by Cumulus.

Recently, a number of startups have taken off with the idea of running
databases -- even classical technology such as Postgres -- in the cloud
(e.g. Greenplum and Aster Data). While several systems aim at OLAP
applications, none of them allows for true online aggregation as Cumulus does,
and none of the systems takes as radical a lightweight, systemsy approach
as Cumulus.
} % end nop

