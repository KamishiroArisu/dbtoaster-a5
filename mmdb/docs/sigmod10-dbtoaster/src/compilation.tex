\section{Recursive Compilation}
\def\Dom{\mathrm{Dom}}
\def\algsum{\mathrm{sum}}
\def\algif{\mathrm{if}}
\def\algthen{\mathrm{then}}
\def\algelse{\mathrm{else}}
\def\algforeach{\mathrm{foreach}}

\newcommand\caret{\mathbin{\char`\^}}
\renewcommand{\algorithmiccomment}[1]{// #1}

\feature{Query Compilation and Processing Overview}
\compiler\ focuses on applications issuing standing queries on a database which
subsequently process continuously changing, large volumes of input tuples.  Our
data model differs from existing stream processors, namely we consider a
database as a set of relations, each subject to an arbitrary sequence of
inserts, updates and deletes. In contrast, stream processors assume a rigid
insertion and deletion semantics, through constructs such as windows, or
punctuations and heartbeats. In \compiler, each tuple has an arbitrary
lifetime. Indeed, many stream applications, such as order book trading and
moving object applications, are self-managing -- the application logic and event
sources ensure that state does not grow unboundedly in practice.  \compiler\
produces imperative procedures to process queries given data manipulation
events. In this section we describe the compilation algorithm to transform
declarative queries into event handlers.

\compiler's recursive compilation algorithm performs its work on queries
represented in a \textit{map} calculus. We start this section with a description
of the map calculus, and subsequently describe the various stages of compilation
as transformations of terms in the calculus, identifying key properties of terms
that hold following each compilation stage.

\subsection{Map Calculus}
\feature{Map terms}
The map calculus consists of the following types of terms:

\[
c 
\quad\;\;
x
\quad\;\;
f+g
\quad\;\;
f*g
\quad\;\;
\algsum_f(Q)
\]

Above $c,x$ denote numerical constants and variables respectively, $f,g$ are map
calculus terms (or simply \textit{maps}) and $\algsum_f(Q)$ denotes a sum
aggregation consisting of a relational calculus formula $Q$.

Variables in maps are either {\em free} or {\em bound}.
Given a map $f$ with free variables $\vec{x}$ (referred to in the order in which
they first appear in $f$), $f[\vec{a}]$, where $\vec{a}$ is a tuple of variables
and constants of the same arity as $\vec{x}$, denotes the substitution of each
$x_i$ in $f$, by $a_i$. The variables $\vec{x}$ in $f[\vec{a}]$ are then called
bound.  For example, the free variables of $5 * x + y$ are $x,y$ and $(5 * x +
y)[z, 2]$ is $5 * z + 2$ with free variable $z$.  The number of free variables
in a map is also called the map's arity.  We occasionally use $f[]$ to indicate
a map term $f$ with all variables bound.
We also define one particular kind of aggregate term, namely a
\textit{constraint-only} aggregate, where the relational part of the aggregation
contains only constraints, and each constraint includes at least one scalar
subaggregate query. Constraint-only aggregates are equivalent to conditionals,
that is:
\begin{align*}
\algsum_{1}(\phi( & <,100,\algsum_{A}(R(A,B)))) =\\
& \algif \; \phi(<,100,\algsum_{A}(R(A,B))) \; \algthen \; 1 \; \algelse \; 0
\end{align*}

\feature{Relational calculus formulae}
Next, we present our relational calculus formulae, which is a simplification of
the domain relational calculus:

\[
R(A,B)
\quad
\{\tuple{a,b}\}
\quad
\phi(\theta,f,g)
\quad
R \wedge S
\quad
R \vee S
\quad
\emptyset
\quad
\{\tuple{}\}
\]

\noindent respectively denoting relations (where attributes are pairs of
variables and domains, denoted $A, \Dom(A)$), tuples, constraints, conjunctions,
disjunctions, the empty set and a constant nullary relation. Here a constraint
is a triple of a comparison operator $\theta \in \{<=,<,=,!=,>,>=\}$, and two
map terms enabling nested queries, denoted $\phi(\theta, f, g)$.  A conjunction
of two relations, $R \wedge S$, corresponds to a natural join between the two
relations, while a disjunction $R \vee S$, represents a union with an implicit
projection (a projection to common variables of $R$ and $S$).
We can express selection operator ($\sigma$) from relational algebra as a
conjunction of a relation and a binary constraint, i.e.  $\sigma_{a<5}(R(A,B))
= R(A,B) \wedge \phi(<,A,5)$.  W.l.o.g, we will use renaming only directly above
the leaves of a calculus formula syntax tree.  We use a multiset semantics for
relations as in SQL; none of the operations of relational calculus eliminate
duplicates.
\todo{We return to the difference operation later.}
\todo{Example SQL query and our representation here?}
\todo{Safe and unsafe variables, and their relationship to bound variables, or
  do we describe this during aggregate simplification?}

\feature{Group-by aggregations}
Our aggregation terms with all variables bound, i.e. $\algsum_{f}(Q)[]$,
correspond to non-grouped aggregation in SQL (with the exception of syntax since
$g[]$ is a scalar value, while SQL aggregates yield a singleton relation).  We
represent SQL group-by aggregates as map terms with free variables corresponding
to group-by attributes. However a map term $g[\vec{a}]$
differs from a SQL group-by aggregate since the function $g$ is defined everywhere.
That is, suppose $g[a] = \algsum_{B}(R(A,B) \wedge \phi(\theta,A,0))[a]$. Here $g[a]$
produces $\{\tuple{0}\}$ rather than $\emptyset$ when
$R(a,B) \wedge \phi(\theta,a,0))$ is the empty relation and
$\algsum_{B}(\emptyset) = 0$, while the SQL aggregate-group-by query
$\sigma_{A=a}$(\texttt{select $A$, sum$(B)$ from $Q$ group by $A$})
gives no value on tuples $\{\tuple{a}\}$ that do not exist in $\Dom(A)$.


\feature{Monomials, polynomials and semirings}
We adopt a canonical representation for map terms and relational calculus
formulae, based on distributivity properties of their operations. We define
basic map terms as constants, variables, and aggregate operations, and the basic
calculus formulae as relations, constraints, an empty and a constant
nullary relation. These are the only elements that can appear as leaves in our
parse trees. Next we observe that conjunctions and disjunctions in our relational
calculus exhibit the same associativity and distributivity properties as the
addition and multiplication operators for map terms. That is:

\[
f*(g+h) = (f*g) + (f*h)
\quad
R \wedge(S \vee T) = (R \wedge S) \vee (S \wedge T)
\]


We refer to conjunctions and multiplications as products for the calculus and
maps respectively, and disjunction and addition as sums.  We can then define a
\textit{monomial} as a product of basic map terms or calculus, and a
\textit{polynomial} as a sum of monomials. We use polynomials of map terms and
calculus formulae for simplicity of compilation -- polynomials provide
separation between the occurrence of sums and products in the parse tree (that
is no product expression can contain a sum operator).  We provide an example of
a map term and calculus formulae, monomials and polynomials in
Figure~\ref{tbl:monpolyex}.

\comment{
\begin{figure}[htbp]
\begin{tabular}{l|c}
Terminology & Notation, example \\
\hline
Map monomial                 & \\
Safe variables               & \\
Unsafe variables             & \\
Delta                        & \\
Basic term                   & \\
Basic calculus element       & \\
Map polynomial               & \\
Relational monomial          & \\
Relational polynomial        & \\
Semiring operations          & \\
Map semiring elements        & \\
Relational semiring elements & \\
Constraint-only aggregate    & \\
\end{tabular}
\end{figure}
}

\begin{figure*}[htbp]
\begin{center}
\begin{tabular}{c|c|c}
           & Map term & Relational calculus formula\\
\hline
Monomial   & $5*x*sum_{f}(Q)$
           & $R(A,B) \wedge S(C,D) \wedge \phi(=,B,C)$\\
Polynomial & $(5*x*sum_{f}(Q_1)) + sum_g(Q_1) $
           & $(R(A,B) \wedge S(C,D) \wedge \phi(=,B,C)) \cup (T(A,D) \wedge U(B,C,D))$
\end{tabular}
\label{tbl:monpolyex}
\caption{Map term and relational calculus, monomial and polynomial examples.}
\end{center}
\end{figure*}

\begin{figure*}
\begin{tabular}{l|c|l|c|c}
Compilation procedure & Method & Description & Example input & Example output \\
\hline
Polynomial construction     & $roly\_poly$ &
& $3 * (\algsum_{1}(R(A,B)) + \algsum_{1}(S(C,D)))$
& $(3*\algsum_{1}(R(A,B))) + (3*\algsum_{1}(S(C,D)))$
\\
Delta transformation        & $delta$ &
&
&
\\
Factorization               & $simplify$ &
&
&
\\
Bound variable propagation  & $simplify$ &
&
&
\\
Aggregate extraction        & $extract\_aggregates$ &
&
&
\\
Bigsum rewriting            & &
&
&
\\
\end{tabular}
\caption{Summary of transformations performed during recursive compilation.}
\label{fig:methodsummary}
\end{figure*}

Given such consideration of monomials and polynomials, we can encapsulate such
distributivity properties or products over sums using semirings of map terms,
and calculus formulae. A semiring is a set of elements, over which both a sum
($+$) and a product ($\times$) operator can be defined, along with two specific
identity elements: a zero and a one element. For map terms, the zero and one
elements correspond to the constants 0 and 1, while for the calculus, these are
the empty set ($\emptyset$) and the constant nullary relation
($\{\tuple{}\}$). The semiring defines simplifications between zero elements and
sums and products, as well as between one elements and products.

\feature{Polynomial construction}
\compiler\ transforms queries into polynomial form by performing one key
simplification, namely factorization of aggregate argument terms over their
corresponding calculus formulae, in addition to recursively transforming
nested map terms in constraints. We now describe this in more detail.

Consider an aggregate term $q[] = \algsum_{f}(Q)$. Following recursively
constructing polynomials of $f$ and $Q$, we can consider $f$ a polynomial map
term with monomials $M_f$, and $Q$ a polynomial relational calculus formula
with monomials $M_Q$. Factorizing involves separating uncorrelated basic
elements of monomials, that is for some term and relational monomial $m_f
\in M_f, m_Q \in M_Q$, we may be able to partition monomial elements (recall
monomials are products of basic elements) into term-relational pairs for simpler
aggregation. The desired term-relational pairs are:

\[\{(l_{fi},l_{Qi}) | l_{fi} \subseteq m_f, l_{Qi} \subseteq m_Q,
vars(l_{fi}) \cap vars(l_{Qi}) \neq \emptyset\}\]

\noindent where $vars(l_{fi}), vars(l_{Qi})$ denotes variables used by sets of basic
map terms and calculus formulae respectively, and $l_{fi} and l_{Qi}$ are minimal
subsets where the property holds. This property states that for any resulting
pair $(l_{fi}, l_{Qi})$, the map term $l_{fi}$ uses variables defined in $l_{Qi}$. We can
then write the resulting factorized term as:
$\algsum_{l_{f1}}(l_{Q1}) + \ldots + \algsum_{l_{fn}}(l_{Qn})$.
We give a simple example of factorization below:

\begin{align*}
\algsum_{A*B*D*E}( & R(A,B,C) \wedge S(D,E,F) =\\
& \algsum_{A*B}(R(A,B,C)) * \algsum_{D*E}(S(D,E,F))
\end{align*}

Here a term and relational monomial pair $(A*B*D*E, R \wedge S)$ is factorized
into a set of term-relational monomials $\{(A*B,R), (D*E,S)\}$. Due to space
constraints, we omit the full details of an algorithm for computing such a
factorization, and simply sketch an outline. Our approach constructs a graph for
each term-relational monomial pair with nodes as variables from each monomial,
and edges fully connecting those variables used in each basic map term or
relational expression. In our example above, we create the following edgelist:
$\{(A,B), (B,C), (C,A), (D,E), (D,F), (E,F)\}$ as defined by the variables from
relations $R,S$. Note that the map term monomial $A*B*D*E$ does not induce edges
since each basic map term is a single variable. Factorization then involves
determining connected components on such a graph.

In general, the construction of polynomials may lead to an exponential number of
basic terms in our maps, however this is often acceptable in practice since
queries tend to be small. This exponential query representation is exhibited in
many other contexts, including in the conversion between disjunctive and
conjunctive normal forms in first-order logic. In summary, our polynomials have
a well-defined structure. The top-level of a parse tree is a map polynomial,
which may include aggregate terms. These aggregate terms include a polynomial
calculus formulae, in addition to a potentially nested aggregate argument map
polynomial. Relational calculus polynomials themselves may include nested map
polynomials, in our constraints.  This nested polynomial structure is
illustrated in Figure [XXX].


\subsection{Query Compilation}
In this section we describe the core compilation mechanisms applied to map and
relational polynomials. Our compilation algorithm proceeds by recursively
applying a basic compilation step. The basic compilation step applies in two
phases, first deriving a \textit{delta} query to an input query, followed by
simplification of the delta query. 

\feature{Delta Query Derivation}
In \compiler, a \textit{delta} refers to an insertion or deletion of a single
tuple to a relation that arrives on an update stream. A delta corresponds to a
binding of a relation's variables to constants at runtime. We denote an
insertion delta to a relation $R(A,B)$ as $\Delta_{+R(a,b)}$ (similarly
$\Delta_{-R(a,b)}$ for deletion), where $\{\tuple{a,b}\}$ describes the values
present in an update stream tuple. As an overview, \compiler\ replaces the
appropriate relations with the new delta's bound variables, and unifies these
bound variables to produce a delta query that is simpler in nature than the
original query.
As a simple example, we can write
\begin{align*}
\Delta_{+R(a,b)} \algsum_{A}(R(A,B) & \wedge S(B,C)) =\\
& \algsum_{a}(S(B,C) \wedge \phi(=,b,B))
\end{align*}
\noindent~where we can see we have eliminated one input relation to the join
with a single tuple, and propagated the bound variables $a,b$ throughout the
term.  We now describe the transformations performed with deltas on both map
terms and calculus formulae in more detail.

\begin{figure}
\begin{align*}
\Delta_{+R(\vec{r})} f \oplus g & := 
    \Delta_{+R(\vec{r})} f \oplus \Delta_{+R(\vec{r})} g \\
\Delta_{+R(\vec{r})} f \otimes g & := 
(\Delta_{+R(\vec{r})}f)\otimes g \oplus
(f \oplus \Delta_{+R(\vec{r})} f)\otimes(\Delta_{+R(\vec{r})}g)\\
\Delta_{+R(\vec{r})}(c) & := 0 \\
\Delta_{+R(\vec{r})}(x) & := 0 \\
\Delta_{+R(\vec{r})} \algsum_{f}(Q) & :=
\begin{cases}
(\algif\; \phi_{i,new} \; \algthen\; \Delta_{+R(\vec{r})}\; f \algelse\; 0)\\
\;\; + (\algif\;
  \left( \phi_{i,new} \wedge \neg \phi_{i,old } \right)
\; \algthen\; f \;\algelse\; 0)\\
\;\; + (\algif\;
\left( \neg \phi_{i,new} \wedge \phi_{i,old} \right) \;
\algthen\; -f \;\algelse\; 0)
\\
\mbox{where }
  \phi_{i,new} = \\
  \qquad
    \bigwedge_i
    \phi_i(\theta_i, g_i+\Delta_{+R(\vec{r})}g_i, h_i + \Delta_{+R(\vec{r})}h_i)
\\
\mbox{and }
  \phi_{i,old} = \bigwedge_i \phi_i(\theta_i, g_i, h_i)
\\
\ldots \mbox{ when $Q = \bigwedge_i \phi_i(\theta_i,g_i,h_i)$}
\\
\\
\algsum_{\Delta_{R(\vec{r})}f}(Q) + \algsum_{f}(\Delta_{+R(\vec{r})} Q)\\
\; \; + \algsum_{\Delta_{+R(\vec{r})}f}(\Delta_{+R(\vec{r})} Q)\\
\ldots \mbox{ otherwise}
\end{cases}
\\
\Delta_{+R(\vec{r})}(S(\vec{A}))         & :=
\begin{cases}
\{\tuple{\vec{r}}\} & \mbox{ if $R = S$}\\
\emptyset           & \mbox{ otherwise}
\end{cases}
\\
\Delta_{+R(\vec{r})}(\phi(\theta, f, g)) & :=
\begin{cases}
\emptyset & \mbox{ $\Delta_{+R(\vec{r})} f = 0, \Delta_{+R(\vec{r})} g = 0$}\\
\bot      & \mbox{ otherwise}
\end{cases}
\\
\Delta_{+R(\vec{r})} \emptyset           & := \emptyset \\
\Delta_{+R(\vec{r})} \{\tuple{\vec{r}}\} & := \emptyset \\
\end{align*}
\label{fig:deltarules}
\caption{Delta transformations of semirings, map terms and relational calculus
  formulae.}
\end{figure}

Figure~\ref{fig:deltarules} presents delta transformations for a semiring, which
applies to both semirings of map terms and relational calculus formulae. The
symbols $\oplus, \otimes$ corresponds to semiring sum and product operators for
both types of expressions. The figure also describes deltas of basic map terms,
which is straightforward for constants and variables, while for sum aggregates,
we sum the combinations of how we may apply deltas to the aggregate argument and
relational part.

The interesting case is for a constraint-only aggregate, whose value can
arbitrarly change despite conjunctive monomials. This is based on the insertion
and removal of tuples in the result set, when considering a constraint-only
aggregate as a conditional. There are three cases:
i) rows whose new value satisfy the condition are
  incremented by the delta query value;
ii) rows whose new value satisfy the condition, but did not previously, must
  be added to the result set (note that these rows are a subset of those
  incremented by the delta query value in the previous case);
iii) rows which previously satisfied the condition but do not any more, must be
  removed from the result set, according to the old value they contribute.
While this condition holds in the general case for arbitrary nested
conditionals, we briefly revisit optimizing this delta rule in the presence of
monotonic aggregates in Section [XXX].
\todo{Give simple example to illustrate this transformation only.}
\todo{Change constraints-only case to match single rule for deltas with five
  branches in the new calculus.}

Finally, we come to deltas for relational calculus formulae. The main point to
note is that the delta for a constraint is only defined for non-nested
constraints, that is when then map term arguments in the constraint have a
zero-valued delta. \compiler\ explicitly flattens queries to ensure this
property, lifting such constraints whenever possible, otherwise transforming to
a constraints-only aggregate, which can then be handled with the appropriate map
term delta transformation.


\feature{Aggregate Simplification}
Following delta query derivation, we simplify the resulting map term by
exploiting variables bound by the delta. In short, delta bound variables, which
are constants supplied from a tuple at runtime, enable constraint
simplification, via variable unification, resulting in the propagation of delta
bound variables throughout maps and relational calculus. The map term
on which no further bound variable propagation may be applied is used for delta
query evaluation. We describe this process in more detail.

Given an aggregate term $\algsum_{f}(r)$, our simplification process applies in
two phases. First we identify equality constraints in the relational part of the
aggregate $r$, constructing a mapping from equated variable pairs by performing
variable unification. This mapping is used to perform variable substitution. Our
unification algorithm accepts a set of identities defining variables that must
be preserved in any mapping (i.e. must not appear in the domain of a
mapping). Indeed, delta bound variables are passed as identities to substitute
as many safe and unsafe variables as possible with these bound
variables.

Unification yields a map term with equality constraints corresponding
to mappings removed. 
For example, unifying the query
$\algsum_{1}(R(A,B) \wedge S(C,D) \wedge \phi(=,B,C))$
with variables $C,D$ as identities yields $\algsum_{1}(R(A,C) \wedge S(C,D))$.
However, requiring variables as identities during
unification can lead to inconsistencies -- a mapping entry that does not
preserve the required identities. Such inconsistencies must remain present in
the resulting map term.
For example, unifying
$\algsum_{1}(R(A,B) \wedge S(C,D) \wedge T(E,F) \wedge \phi(=,B,C) \wedge
\phi(=,D,E)$ with $C,D,E,F$ as identities yields
 $\algsum_{1}(R(A,C) \wedge S(C,D) \wedge T(E,F) \wedge \phi(=,D,E))$.

The mapping obtained from the relational part of the aggregate can then be used
to perform substitutions in the aggregate argument map term. We recursively
apply this simplification to any nested aggregates in constraint map terms.
At a high level, this form of aggregate simplification induces a structured
propagation of delta bound variables. Delta bound variables are propagated
top-down through a parse tree as a result of recursive application, first being
passed to the relational part of aggregates to replace safe and unsafe
variables. Subsequently aggregate simplification propagates delta bound
variables into the aggregate argument term, essentially a sideways propagation
within sibling parse trees present at aggregates.
\todo{Show parse tree of bound variable propagation.}


\feature{Recursive compilation}
Together, the delta query derivation and aggregate simplification form one step
of our recursive compilation algorithm that is applied for a delta to a single
relation. 
Algorithm \ref{alg:compile} shows our algorithm for recursively compiling deltas
to produce source code. This is a simplified algorithm that does not present our
handling of nested queries. We return to this in the following section.
Our recursive compilation algorithm works by starting with a flattened
query, compiling deltas for all relations. This results in a set of simplified
delta map terms which we can evaluate to compute the query result. Note that
each delta map term contains one fewer input relation, replaced with bound
variables. We begin a recursive compilation step with our delta map
terms as the new starting point for compilation. Thus by compiling again for
all relations, we obtain deltas of delta map terms, and repeat the process until
the delta map terms produced contain no relations, rather, all relations'
variables have been replaced by bound variables from deltas.

\todo{Comment on renaming of bound variables in a map term to a free variable
 during map construction.}

\todo{Note on $k$-level compilation, and using naive compiler to do the rest
  after $k$}

\todo{Briefly describe aggregate extraction.}

\newcommand{\indentcode}{\hspace{5mm}}
\begin{algorithm}
\caption{compile($db\_schema, map, params, term$)}
\label{alg:compile}
\begin{algorithmic}[1]
\STATE $code \leftarrow []$
\FOR{$(relation, schema) \in db\_schema$}
  \STATE \COMMENT{Compile delta for a single relation event}
  \STATE $bound\_vars \leftarrow prefix("b\_\_", schema) \cup ext\_vars$
  \STATE $dt \leftarrow filter(\lambda x. x \neq 0, simplify($
    \STATE \indentcode $delta(ext\_maps,term),bound\_vars, params)$
  \STATE $name\_prefix \leftarrow map+relation$
  \STATE $(done, todos) \leftarrow extract\_aggregates($
    \STATE \indentcode $dt, name\_prefix, bound\_vars)$
  \FOR{$(done\_terms, done\_params) \in done$}
     \STATE $code += generate\_code(relation,$
     \STATE \indentcode $map, done\_terms, done\_params)$
  \ENDFOR
  \STATE \COMMENT{Recur for remaining terms}
  \FOR{$(next\_map, next\_params, next\_term) \in todo$}
  \STATE $code += compile(db\_schema, $
  \STATE \indentcode $next\_map, next\_params, next\_term)$
  \ENDFOR
\ENDFOR
\RETURN $code$
\end{algorithmic}
\end{algorithm}

\begin{example}\feature{Flat query compilation example}
We now walk through compiling the following flat query, given in SQL as
\texttt{select sum(a*d) from R join S join T}, and in our representation as:
\[\algsum_{A*D}(R(A,B) \wedge S(B,C) \wedge T(C,D))\]
\begin{align*}
& q[] = \algsum_{A*D}(R(A,B) \wedge S(B,C) \wedge T(C,D))\\
& \hspace{-1.5cm}\mbox{\sc Compilation level 1:}\\
\Delta_{+R(a,b)} & q[] +=
a * \underbrace{\algsum_{D}(S(B,C) \wedge T(C,D) \wedge \phi(=,b,B)}_{qR[B']}\\
\Delta_{+S(b,c)} & q[] +=
\underbrace{\algsum_{A}(R(A,B) \wedge \phi(=,B,b))}_{qS1[B']} * \\
& \underbrace{\algsum_{D}(T(C,D) \wedge \phi(=,c,C))}_{qS2[C']}\\
\Delta_{+T(c,d)} & q[] +=
d * \underbrace{\algsum_{A}(R(A,B) \wedge S(B,C) \wedge \phi(=,c,C)}_{qT[C']}\\
\\
& \hspace{-1.5cm}\mbox{\sc Compilation level 2:}\\
\Delta_{+S(b,c)} & \algforeach \; B': qR[B'] += \\
& \algsum_{D}(T(C,D) \wedge \phi(=,B',b) \wedge \phi(=,C,c))
 \\
\therefore \Delta_{+S(b,c)} & qR[b] +=
\underbrace{\algsum(T(C,D) \wedge \phi(=,C,c))}_{qRS[C'] = qS2[C']}
\\
\Delta_{+T(c,d)} & \algforeach \; B': qR[B'] +=  \\
& d * \underbrace{\algsum_{1}(S(B,C) \wedge \phi(=,B',B) \wedge
  \phi(=,C,c))}_{qRT[B'',C']}
\\
\Delta_{+R(a,b)} & qS1[B'] = \algforeach \; B': qS1[B'] +=
a * \algsum_{1}(\phi(=,b,B'))
\\
\therefore \Delta_{+R(a,b)} & qS1[b] += a \\
\Delta_{+T(c,d)} & qS1[B'] = 0 \\
\Delta_{+R(a,b)} & qS2[C'] = 0 \\
\Delta_{+T(c,d)} & qS2[C'] = \algforeach \; C': qS2[C'] +=
d * \algsum_{1}(\phi(=,C',c))
\\
\therefore \Delta_{+T(c,d)} & qS2[c] += d\\
\Delta_{+R(a,b)} & \algforeach \; C': qT[C'] += \\
& a * \underbrace{\algsum_{1}(S(B,C) \wedge \phi(=,b,B) \wedge
  \phi(=,C,C'))}_{qTR[B',C''] = qRT[B'',C']}
\\
\Delta_{+S(c,d)} & \algforeach \; C': qT[C'] += \\
& \algsum_{A}(R(A,B) \wedge \phi(=,B,b) \wedge \phi(=,c,C'))\\
\therefore \Delta_{+S(b,c)} & qT[c] +=
\underbrace{\algsum_{A}(R(A,B) \wedge \phi(=,B,b))}_{qTS[B'] = qS1[B']}
\\
& \hspace{-1.5cm}\mbox{\sc Compilation level 3:}\\
\Delta_{+S(b,c)} & \algforeach B',C': qRT[B',C'] += \\
& \algsum_{1}(\phi(=,B',b) \wedge \phi(=,C',c))
\\
\therefore \Delta_{+S(b,c)} & qRT[b,c] += 1
\end{align*}
\end{example}
\todo{Add starting equations for each term with tuple in place of relation.}

\noindent The resulting code is:

\begin{verbatim}
int q;
map<int, int> qR, qS1, qS2, qT;
map<tuple<int, int>, int> qRT;

void on_insert_R(int a, int b) {
    q += a*qR[b]; qS1[b] += a;
    foreach c in keys(qT): qT[c] += a * qRT[b,c];
}

void on_insert_S(int b, int c) {
    q += qS1[b] * qS2[c];
    qR[b] += qS2[c]; qT[c] += qS1[b];
    qRT[b,c] += 1;
}

void on_insert_T(int c, int d) {
    q += d*qT[c]; qS2[c] += d;
    foreach b in keys(qR): qR[b] += d * qRT[b,c];
}
\end{verbatim}

We omit the code produced for handling deletions on the input stream -- the code
is analagous to the insertion code seen above with decrements to maps instead of
increments.


\subsection{Nested query compilation}
Throughout this section, we use a simpler form of the VWAP query on orderbooks
as our example of a nested query. The SQL representation of this query is as
follows:

\begin{verbatim}
select sum(P*V) from Bids(P,V)
    where 1000 >
    (select sum(V2) from Bids(P2,V2) where P2 > P)
\end{verbatim}

and the \compiler\ calculus representation:
\begin{align*}
\algsum_{P*V}(& Bids(P,V) \wedge \\
& \phi(>, 1000, \algsum_{V2}(Bids(P2,V2) \wedge \phi(>,P2,P))))
\end{align*}

\feature{Nested variable extraction and flattening}
\todo{
Explain why we flatten, i.e. why there is no delta rule for constraints
with non-zero delta terms, e.g. }
\comment{
\begin{align*}
\Delta_{+R(\vec{r})}\phi(\theta,f,g) & := &\\
&& \algif \phi_{new} \;\algthen\; ??? \; \algelse \; 0 +\\
&& \algif \phi_{new} + \neg \phi_{old} \; \algthen \; ??? \; \algelse \; 0 +\\
&& \algif \neg \phi_{new} + \phi_{old} \; \algthen \; ??? \; \algelse \; 0
\end{align*}

This seems to be because of no clean way to express the delta purely in terms of
parse subtrees, from the constraint -- we need to refer to parent tree nodes and
sibling nodes in the \texttt{then} part of the conditional. What if we had taken
an algebraic approach, where we have a selection operator $\sigma$? We could
then write:
\begin{align*}
\Delta_{+R(\vec{r})} \sigma_{f \theta 0}(Q) & := &
   \sigma_{f_{new} \theta 0}(\Delta_{+R(\vec{r})} Q) \cup \\
&& \sigma_{f_{new} \theta 0 \wedge \neg(f \theta 0)}(Q) - \\
&& \sigma_{not(f_{new} \theta 0) \wedge f \theta 0}(Q)
\end{align*}

The other major reason seems to be no clean separation of selections from unions
and joins, as needed by our semiring abstraction. We could have selections (and
even projections) at the top-level, and then polynomials of unions and joins as
we currently do. However, we cannot have selections at the leaves alone, due to
join predicates for example.
}

In the delta rules we presented above, there is no delta transformation defined
in the case where constraints contain map terms with non-zero deltas. This
occurs with scalar subaggregate queries, and in this section we describe a
transformation that \textit{flattens} out such queries when they are provided by
the user, into a constraint-only aggregate. The delta transformation for a
constraint-only aggregate may then apply to correctly derive a delta query for
a nested query. Note that this flattening concerns nested queries within
constraints -- nested queries that occur as aggregate arguments are directly
handled in the delta rules described above.

Our flattening transformation is essentially a parse tree rotation operation
that separates the usage of variables in a nested fashion from their originating
relation. Flattening replacing nested variables in constraints with fresh
variables called \textit{\bigsum} variables.
\todo{Think of a better name for \bigsum.}
Note the domains of these \bigsum\ variables are also decoupled from the
original nested variables' domains, and we explicitly maintain \bigsum\ variable
domains as discussed in the next section.  Any constraint in the resulting map
term uses only those variables defined within the constraint's map terms, or
\bigsum\ variables. Flattening extracts these nested constraints into a
constraints-only aggregate, with an aggregate argument term consisting of the
remainder of the original map term. Thus our flattening transformation evaluates
the nested constraint for each value of the \bigsum\ variable, and provided this
condition holds, subsequently evaluates the original map term with an additional
constraint between the original nested variable and the \bigsum\ variable. We
present an example of this transformation below.

\begin{align*}
\algsum_{P*V}( & \underbrace{B(P,V)}_{r} \wedge \\
& \underbrace{\phi(<, 1000, \algsum_{V2}(B(P2,V2) \wedge \phi(>, P2, P)))}_{c}) \\
= & \; \algsum_{f}(\phi(<, 1000, \algsum_{V2}(B(P2,V2) \wedge \phi(>, P2, \xi)))) \\
\mbox{ where } f = & \; \algsum_{P*V}(B(P,V) \wedge \phi(=,P,\xi))
\end{align*}

Above, $\xi$ is the \bigsum\ variable introduced for the nested usage of $P$.
The general form of our transformation is:

\begin{align*}
\algsum_{f}( & Q(\vec{A}) \wedge \bigwedge_i \phi_i(\theta_i, g_i[\vec{A}], h_i[\vec{A}])) \\
= & \; \algsum_{
\underbrace{\algsum_{f}(Q(\vec{A}) \wedge \bigwedge_i \phi(=,A_i,\xi_i))}_{t}}(
\underbrace{\bigwedge_i \phi_i(\theta_i, g_i[\vec{A} \mapsto \vec{\xi}], h_i[\vec{A} \mapsto \vec{\xi}])}_{c})
\end{align*}

\noindent where $g[\vec{A} \mapsto \vec{xi}]$ denotes the substitution of
variables $\vec{A}$ by variables $\vec{xi}$ in the map term $g$. 

\feature{Domain maintenance}
Given our flattening transformation described above, our recursive compilation
may instantiate maps whose keys include \bigsum\ variables. Since \bigsum\
variables are not present in any delta tuple, we must maintain the domains of
these variables based on the expansion and contraction of the original nested
variable domain (which does appear in a delta). Furthermore, we must also
maintain maps with \bigsum\ variables as keys given changes in the nested
variable domains. In general, map term evaluation with unseen values of \bigsum\
variables cannot be performed incrementally with delta map terms. Consider a map
$m$ defined as the following map term:

\[\algsum_{A}(R(A,B)\wedge \phi(<, 0, \xi))\]

\todo{Better example.}

Here \bigsum\ variable $\xi$ is a key for the map $m$ since it is free in the
map term.
Now, suppose we see a new value of the nested variable (in a delta
$\Delta_{+R(\vec{r})}$), corresponding to $\xi$, and that this value is not in
the domain of $\xi$. For map $m$, we do not know the value of $m[\xi]$ prior to
the update (we cannot maintain a map on every possible key value), furthermore,
such a value is not the initial value of the aggregate. Hence we are unable to
use a delta map term to handle the delta, and must compute the map term from
scratch. In \compiler, we currently perform this with standard query processing
techniques, namely relational query plans translated directly to C++ code, as
opposed to processing deltas. In certain cases, we can avoid this from-scratch
computation by leveraging neighboring entries in the map, for example in the
presence of monotonic map terms. We return to these optimizations in Section
[XXX].


\begin{example}\feature{Nested query compilation example}
We present a compilation trace for the original VWAP query (not the simplified
form shown above). First we present the flattened form:
\begin{align*}
q[] & = \sum_{bigsum\_p}{\Big[\algif \; \phi(>, f, g) }\\
    & \algthen \; \algsum_{P*V}(Bids(P,V) \wedge bigsum\_p = P) \;
      \algelse \; 0 \Big]\\
\mbox{where } & f = 0.25*\algsum_{V1}(Bids(P1,V1))\\
\mbox{and } & g = \algsum_{V2}(Bids(P2,V2) \wedge P2 > bigsum\_p)
\\
& \hspace{-1.75cm}\mbox{\sc Compilation level 1:}\\
\Delta_{+Bids(p,v)} & q[] += \sum_{bigsum\_p} \Big[ \\
& [ \algif \; \phi(>,\underbrace{f}_{m1[]}+\Delta f, \underbrace{g}_{m2[bigsum\_p]}+\Delta g) \; \\
& \algthen \; \Delta \algsum_{P*V}(Bids(P,V) \wedge bigsum\_p = P) \;
    \algelse \; 0 ] \\
+ \; & [ \algif \; \phi(>,\underline{f}+\Delta f, \underline{g}+\Delta g) \wedge
    \neg\phi(>,\underline{f},\underline{g}) \\
& \algthen \; \underbrace{\algsum_{P*V}(Bids(P,V) \wedge
    bigsum\_p = P)}_{m3[bigsum\_p]} \;
    \algelse \; 0 ] \\
+ \; & [ \algif \; \neg \phi(>, \underline{f}+\Delta f, \underline{g}+\Delta g)
    \wedge \phi(>,\underline{f},\underline{g}) ] \\
& \algthen \; - \underline{\algsum_{P*V}(Bids(P,V) \wedge bigsum\_p = P)} \;
    \algelse \; 0
\Big]\\
& \hspace{-1.75cm}\mbox{\sc Compilation level 2:}\\
\Delta_{+Bids(p,v)} \; & m1[] += v\\
\Delta_{+Bids(p,v)} \; & \algforeach \; bigsum\_p : \; m2[bigsum\_p] += \\
& \qquad \algif \; p > bigsum\_p \;\algthen \; v \; \algelse \; 0 \\
\Delta_{+Bids(p,v)} \; & \algforeach \; m3[bigsum\_p] += \\
& \qquad \algif \; bigsum\_p = p \; \algthen \; p*v \; \algelse \; 0\\
\therefore \; & m3[p] += p*v
\end{align*}
\end{example}

Above, the underlined terms correspond to duplicate map terms, where maps can be
reused. The code produced is:

\begin{verbatim}
int q, m1;
map<int, int> m2, m3;

void on_insert_bids(int p, int v) {
  // Initial value computation
  if ( !m2.haskey(p) )
    foreach b_p, b_v in bids: m2[p] += (p < b_p? b_v : 0);

  if ( !m3.haskey(p) ) {
    foreach b_p, b_v in bids:
      m3[] += (p == b_p? b_p*b_v : 0);
  }

  // Compilation level 1
  foreach bigsum_p in dom_bigsum_p:
    q += if (m1+0.25*v > m2[bigsum_p]+(p>bigsum_p? v : 0))
         then (p==bigsum_p? p*v : 0) else 0
    q += if (m1+0.25*v > m2[bigsum_p]+(p>bigsum_p? v : 0))
           && (m1 <= m2[bigsum_p])
         then m3[bigsum_p] else 0
    q += if (m1+0.25*v <= m2[bigsum_p]+(p>bigsum_p? v : 0))
           && (m1 > m2[bigsum_p])
         then -m3[bigsum_p] else 0

  // Compilation level 2
  m1 += 0.25*v;
  foreach bigsum_p in keys(m2):
    m2[bigsum_p] += (p > bigsum_p? v : 0);
  m3[p] += p*v;
}
\end{verbatim}

\comment{
\subsection{Code Generation}
Describe:
\begin{itemize}
\item accumulator language which allows simple code generation for
procedural languages such as C++/Java.
\item differences between simple accumulator language and implementations of
relational calculus which pass around sets, resulting in a larger greater
copying of data and more memory pressure. Accumulator language is even different
to pipelining query plans, which can push tuples through to an accumulator
(i.e. incr() function of an aggregate) at a tuple granularity, but still copies
data from using an operator processing model.
\end{itemize}
}