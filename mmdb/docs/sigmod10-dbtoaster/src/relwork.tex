\section{Related Work}

To the best of our knowledge, there has been limited work addressing the question
of how to compile SQL queries to a low-level imperative language such as C++.
With this in mind, in this section we contrast our map algebra, query rewriting
and execution to several more extensively investigated topics, namely view
maintenance, main-memory databases as well as stream and event processing.

\textbf{View maintenance.}
View maintenance algorithms are in abundance in the database literature, with
topics ranging from efficient view maintenance algorithms given base relation
deltas~\cite{colby-sigmod:96}, which may be eagerly or lazily
applied~\cite{yan-vldb:95,zhou-vldb:07}, to the question of which views to
actually materialize and how to use such views during query
optimization~\cite{kotidis-tods:01,zhou-icde:07}. Zhou et al.~\cite{zhou-vldb:07}
describe a lazy view materialization technique, describing their computation of
deltas starting from a similar point to ours, namely that where a single tuple
has changed as part of a join tree input. However, their approach does not
generalize to the same extent as our map algebra, particularly in the case of
nested aggregations such as the VWAP query seen in this work. Palpanas et
al.~\cite{palpanas-vldb:02} consider an extension of view maintenance algorithms
to handle non-distributive aggregates, using a selective recomputation strategy
to update only those groups affected by new tuples. More pertinent to the
main-memory database context, Roussopoulos~\cite{roussopoulos-tods:91} presents a
pointer-based (also referred to as an index) approach to implement views with
ViewCache, which incrementally maintains views using algorithms derived from
relational operations. Griffin and Libkin~\cite{griffin-sigmod:95} study the
problem of incremental view maintenance for relations with duplicates, using a
bag algebra for equational reasoning about query operators and to derive update
rules.

Our map expression algebra differs from the issue of view maintenance, in that it
computes a set of maps to support incremental maintenance of a query result,
rather than simply maintaining a single view.
\comment{
Our approach to selecting maps under a memory constraint differs from
traditional cost models to choose views to materialize, given our top-down approach to
decomposing queries with maps.}
Furthermore, given our target application of non-interactive queries and a static
query workload, we need not maintain base relations, unlike standard relational
query processors which do so to provide ad-hoc query capability at the expense of
orders of magnitude performance as seen in our experimental section. This
obviates the need for any pointer or index-based approach to track back to
original base relation tuples, unlike existing main-memory database systems. Most
critically, the notion of compiling these view maintenance algorithms is clearly
a simple and effective technique for generating query executors for unmatched
performance.

\comment{
\begin{itemize}
  \item Asymmetric increment techniques \cite{yang-icde:05}.
  \item Group-by optimizations \cite{yan-icde:94}.
\end{itemize}
\noindent \textbf{DataCubes}
\begin{itemize}
  \item Classical literature \cite{gray-icde:96,mumick-sigmod:97}.
  \item TODO: more papers if we head into cubes.
\end{itemize}
}

\noindent \textbf{Stream and event processing.}
Our work can loosely be compared to complex event and stream processing
engines~\cite{wu-sigmod:06,agrawal-sigmod:08,white-pods:07,motwani-cidr:03,abadi-vldbj:03}
due to the common goal of handling frequently changing data.
SASE+~\cite{agrawal-sigmod:08} and Cayuga~\cite{white-pods:07} are complex event
processors that focus on sequence and pattern query processing on event streams
using NFA-based approaches to process a variety of patterns including Kleene
closures and negation operators. Relational stream processing engines such as
STREAM~\cite{motwani-cidr:03} and Aurora~\cite{abadi-vldbj:03} investigate
continuous query processing architectures that are capable of evaluating
stream-equivalent versions of the standard relation algebra. From the systems
perspective, several novel techniques were developed to address the low-latency,
high-throughput needs of stream applications, including scheduling, load
shedding, and approximate query processing.
Many of these systems are simultaneously addressing the issue of efficiency and
the semantic requirements of their target applications, exploiting such semantic
properties to attain performance. In contrast, our approach is not to reinvent
the wheel in terms of data and query models -- we focus on standard relational
algebra, and an intuitive extension via snapshots and temporal logic, and purely
take on the challenge of designing query executors to best suit a main-memory
environment, arguing for an alternative to plan-based execution.

We briefly comment on two other recent works on data stream processing, starting
with the SPADE~\cite{gedik-sigmod:08} language designed for IBM's System S stream
processing engine. The goal of this work is to present the SPADE as a language
for declaratively specifying data stream applications, however the authors do
also discuss their implementation of the language including their compilation
process. The authors discuss compiler optimizations such as operator grouping for
execution across multiple hardware contexts, and operator fusion to reduce
internal queueing. Note that \compiler\ completely eliminates any internal
queueing, since we produce a single-straight line function, and automatically
derives appropriate data structures to support query processing via a single
function. This latter issue is not considered in SPADE. However, \compiler\ does
not support distributed execution in its current form and this is a topic of
future work. Finally, StreamBase~\cite{streambase}, a commercialization of the
Aurora stream processing engine has also studied the problem of compiling stream
programs, however there is little documentation~\cite{sb-patent} regarding the
technical details of their techniques. To the best of our knowledge, their
compilation process does not include generating straight-line processing
functions in the same aggressive manner as \compiler.

 \comment{
 TODO:
contrast our work to the above -- we avoid the need for windows or punctuations
by making adopting an insert/delete/update stream model, and assuming that the
base relations always fit in memory (i.e. \#deletes/updates is proportional to
\#inserts). Stream processors still evaluate an operator-centric query plan,
hence they cannot exploit compiler optimizations on a query processing kernel
function. They also do not deal with maintaining precomputed partial results as
we do with our map decompositions. What can we say about complex event processors
and the NFAs they evaluate?


\begin{itemize}
  \item STRIP: rule-based maintenance of derived data, with finance app example \cite{adelberg-sigmod:97}
\end{itemize}
}


\noindent \textbf{Compiling high-level languages for embedded systems.}
The embedded systems community has partially studied on compiling high-level
languages into a low-level imperative target language.
Newton et al.~\cite{newton-lctes:08} present WaveScript, a scripting language
for processing windows, or segments, of a data stream, and describe a compiler to
construct stream dataflow graphs that are capable of running on XScale CPUs.
WaveScript programs are transformed into dataflow graphs, which may in turn be
manipulated by algebraic rewrite rules, similar to a query optimizer. However,
WaveScope does not address relational queries.
Toman and Weddell~\cite{toman-dbtel:01} describe the DEMO system which compiles
SQL queries into Java or C code, implementing query plans as navigational plans
that utilize pointer access in place of scans and indexing, and supporting query
functionality over existing data structures for an embedded program. The authors
use integrity constraints in their compiler to describe the relationship between
schemas and the physical data structures used by an embedded program, and in turn
generate navigational plans and iterators, which can be used to produce C code in
a straightforward manner. DEMO has taken a theoretical approach to describe
their compilation, and does not consider systems issues nor have they released
a prototype and provided an effective tool to perform compilation.


\noindent \textbf{Main memory databases.}
Early work on main-memory databases studied how to reduce the bottleneck effect
of a variety of I/O tasks including recovery tasks such as checkpointing, as well
as logging and locking structures \cite{bohannon-sigmod:99}. There have been
several recent efforts on this topic given the developments in main memory.
Boncz et al.~\cite{boncz-cidr:05} present the MonetDB/X100 database engine, a
high-performance column-oriented database, that leverages techniques such as
vectorized processing and loop pipelining on chunks of columns while evaluating
operators in query plans. Kallman et al.~\cite{kallman-pvldb:08} describe
H-Store, a shared-nothing distributed main memory database to provide high
throughput processing on OLTP workloads. H-Store achieves its performance through
careful deployment of horizontally partitioning of relations, and localizing OLTP
transactions to partitions. Raman et al.~\cite{raman-icde:08} describe Blink, a
row-oriented main memory query processor that extensively uses compression on
denormalized relations, and applies aggregates while scanning these relations as
its main query processing functionality.

These systems span a wide variety of use cases and focus at rearchitecting at a
lower-level than our approach. For now we have looked at applying a semantic
redesign to describe queries as tuple functions, and reuse much of the work from
the compiler community to provide efficient tuple function execution. Having
applied this first phase to building our runtime, we can now proceed to the
systems level aspects of our tuple functions, for example investigating cache
performance in much greater depth, and in particular understanding which of the
above techniques and optimizations can be applied in our context, as well as
leveraging relevant work from the compiler and programming languages
communities.

\todo{Paragraph for Don Batory's work on compiling DBMSs.}